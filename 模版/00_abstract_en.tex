% English Abstract
\begin{abstractEN}
While algorithmic recommendation systems optimize user experience, they also trigger concerns about information cocoons and attention monopolization. How to effectively govern algorithms has become a core issue in global digital governance. This study leverages the quasi-natural experiment of China's Qinglang Action algorithm governance policy implemented on November 12, 2024, employing interrupted time series (ITS) combined with difference-in-differences (DID) methods. Based on high-frequency data covering 608 days from February 2024 to October 2025 with over 253,227 observations from Weibo's trending list, we systematically evaluate the restructuring effects of algorithm governance on platform ecosystems.

We find: First, algorithm governance significantly promotes attention distribution decentralization, with Shannon entropy immediately increasing by 18.7\% (level coefficient +0.146, $p<0.001$) and HHI decreasing by 11.8\% (level coefficient $-0.0292$, $p<0.001$). The platform breaks the head monopoly pattern by limiting single-topic repeated listings (down 23.4\%) and continuous listing duration (shortened by 31.7\%). Second, risk-based tiered governance leads to content quality differentiation. Social topics' information gap index decreases by 0.561 at level ($p=0.036$), with official media showing a net decrease of 0.0205 ($p<0.001$) relative to non-official media in information gap, while entertainment topics show no significant quality changes. Third, under zero-sum attention constraints, regulation triggers systematic attention reallocation from social to entertainment content. Social topics' share immediately declines by 3.16 percentage points ($p<0.001$) with continuing downward trends, while entertainment rises by 3.84 percentage points (celebrity +3.61pp, gaming +1.12pp). Weekly total heat discontinuity tests verify total conservation ($p=0.63$), exhibiting a pronounced ``waterbed effect.'' Fourth, attention density significantly differentiates across categories, with social topics showing ``fewer but refined'' (average listing duration per topic significantly increases) and entertainment core categories showing ``more and prolonged'' (total listing duration, time share, and average listing duration all significantly increase).

This study proposes theoretical mechanisms of ``meta-organization of meta-organizations'' and ``waterbed effect,'' extending platform governance theory boundaries. The research reveals complex effects of algorithm governance: while regulation successfully achieves diversity improvement and social content quality enhancement, it triggers unintended attention transfer under zero-sum attention constraints. The large-scale reallocation from social to entertainment content may weaken public issue exposure, creating tension between ``regulatory intent'' and ``ecosystem evolution.'' The study provides new evidence for understanding systemic consequences and unintended effects of algorithm governance, offering important policy implications for improving tiered governance systems and constructing a systematic framework of ``quality incentives + dynamic adjustment + cross-platform coordination + user empowerment.''
\end{abstractEN}

\keywordsEN{Algorithm governance; Platform ecosystems; Attention economy; Interrupted time series; Waterbed effect}

\noindent JEL Classification: D83; L51; L86
