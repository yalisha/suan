% English Abstract
\begin{abstractEN}
As the core information allocation mechanism of digital platforms, algorithmic recommendation systems optimize user experience while also triggering public governance challenges such as information cocoons and attention monopolization. How to effectively regulate the negative externalities of algorithms while unleashing their value has become a central issue in global digital governance. This study leverages the quasi-natural experiment of China's Qinglang Action algorithm governance policy implemented in November 2024, employing an identification strategy combining interrupted time series (ITS) with difference-in-differences (DID). Based on high-frequency data from Weibo's trending list, we systematically evaluate the restructuring effects of algorithm governance on platform content ecosystems.

We find that algorithm governance effectively promotes the decentralization of attention distribution, dismantling the monopoly pattern of head topics. Risk-based tiered governance leads to significant quality improvement in social content while entertainment content remains stable, forming differentiated quality improvement pathways. However, under zero-sum attention constraints where total attention is conserved, intensified regulation of high-risk domains triggers systematic attention transfer toward low-risk domains, exhibiting a pronounced ``waterbed effect.'' At the density level, social content shows a ``fewer but refined'' adjustment, while entertainment content shows ``more and prolonged'' expansion. This study reveals the complex effects of algorithm governance: positive outcomes of diversity enhancement and quality improvement coexist with unintended consequences of attention reallocation, creating tension between ``regulatory intent'' and ``ecosystem evolution.'' The research provides new empirical evidence for understanding the systemic consequences of algorithm governance and offers important policy implications for improving tiered governance systems and exploring a systematic framework of ``incentivizing quality content + dynamic monitoring and adjustment + cross-platform coordinated governance.''
\end{abstractEN}

\keywordsEN{Algorithm governance; Platform ecosystems; Attention economy; Interrupted time series; Waterbed effect}

\noindent JEL Classification: D83; L51; L86
