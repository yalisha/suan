\section{制度背景、文献回顾与研究假设}

\subsection{理论基础:注意力作为稀缺资源的平台配置逻辑}

\subsubsection{注意力的零和性质与竞争机制}

在数字平台生态系统中,注意力构成了比信息本身更为稀缺的核心资源\citep{OestreicherSinger2012}。当内容供给呈指数级增长时,用户的认知带宽却受制于生理和时间约束而保持刚性。根据新浪微博2023年财报,用户日均使用时长约52分钟,其中热搜浏览时长约15--20分钟,这一有限的时间窗口构成了注意力资源的刚性约束。

注意力资源具有三个决定性特征。第一,稀缺性:用户日均使用时长存在上限,短期内缺乏弹性\citep{Hosanagar2014}。第二,竞争性:注意力分配给某一话题必然减少其他话题的份额\citep{Berman2018}。在任意时刻,热搜榜只显示50个固定位置,某一话题进入榜单必然挤出其他话题,形成典型的零和博弈格局。第三,不可再生性:注意力一旦消耗无法恢复,用户在某条娱乐话题上停留的时间永久性地从社会类话题的潜在配额中流失。

这种竞争关系在社交媒体平台上表现为注意力拥挤效应:当信息供给增加时,单位信息获得的注意力边际递减\citep{Gelper2018}。设平台上有$N$条话题竞争用户注意力,用户总注意力预算为$\bar{A}$,则第$i$条话题获得的注意力$A_i$满足注意力守恒约束:
\begin{equation}
\sum_{i=1}^{N} A_i = \bar{A}
\end{equation}

这一效应在热搜榜体系中表现为榜内与榜外话题的阅读量呈数量级差异,榜单位置的瞬时稀缺性使得话题间竞争异常激烈。更为关键的是,热搜榜不仅是注意力的消费场所,更是注意力的生产场所。平台通过算法排序和榜单展示,将分散的用户浏览行为转化为可度量、可累积的社会关注度\citep{Lorenz2019}。上榜本身构成一种注意力增值过程,话题进入榜单后获得的注意力呈指数级增长,形成正反馈循环。因此,热搜榜单的任何调整,都是对稀缺注意力资源分配权力的重新定义。

\subsubsection{平台的约束优化模型}

基于注意力零和博弈的底层逻辑,平台内容分发系统可被抽象为一个多目标约束优化问题。传统上,社交媒体平台以最大化用户参与度为首要目标\citep{Kleinberg2021, Hagiu2015}。设平台上第$i$类内容获得的注意力为$A_i$,其产生的用户参与度为$R_i(A_i)$,平台的目标函数可表示为:
\begin{equation}
\max_{\{A_i\}} \quad \sum_i R_i(A_i)
\end{equation}
其中,$R_i(A_i)$通常为凹函数,体现了用户参与度的边际递减特征\citep{He2022}。在监管介入前,平台仅面临基本的资源约束$\sum_i A_i = \bar{A}$,这一优化问题的解往往导致注意力高度集中于少数头部内容,因为头部内容通常具有更高的边际参与度$R_i'(A_i)$。

算法治理行动根本性地改变了平台的约束集合。\citet{Huber2013}指出,外部监管通过引入新的治理约束,改变了平台生态系统价值创造与治理成本之间的权衡。具体而言,《算法专项治理清单指引》引入了三类新约束:

第一,多样性下限约束。设$D(\{A_i\})$为衡量注意力分布多样性的指标(如Shannon熵),监管要求:
\begin{equation}
D(\{A_i\}) \geq D_{\min}^{\text{post}} > D_{\min}^{\text{pre}}
\end{equation}
这一约束直接限制了头部内容的注意力集中度,迫使平台为长尾内容预留基础曝光配额。\citet{Song2018}的研究表明,平台治理机制的动态调整往往是为了满足外部利益相关者施加的约束,而非自发的优化行为。

第二,内容质量约束。设$C_i(Q_i, R_i)$为维持第$i$类内容质量水平$Q_i$的治理成本,该成本函数依赖于内容的风险等级$R_i$。监管要求:
\begin{equation}
Q_i \geq Q_i^{\min}(R_i), \quad \frac{\partial Q_i^{\min}}{\partial R_i} > 0
\end{equation}
对于风险较高的社会类内容,$Q_i^{\min}$显著提升,相应的审核成本$C_i$呈超线性增长\citep{Shin2021}。

第三,透明度与可追溯性约束。监管要求平台留存榜单日志、公示排序机制,实质上增加了平台的合规运营成本。这些新增约束将平台的优化问题转化为:
\begin{equation}
\max_{\{A_i, Q_i\}} \quad \sum_i R_i(A_i) - \sum_i C_i(Q_i, R_i)
\end{equation}
\begin{equation}
\text{s.t.} \quad \sum_i A_i = \bar{A}, \quad D(\{A_i\}) \geq D_{\min}^{\text{post}}, \quad Q_i \geq Q_i^{\min}(R_i)
\end{equation}
这一约束优化框架揭示了算法治理的核心机制:通过缩小平台的可行域,监管改变了最优解的结构,进而触发注意力资源的系统性重新配置。

\subsection{研究假说}

\subsubsection{假说H1:注意力去中心化效应}

\textbf{1. 数学推导}

\textit{命题1.1(可行域收缩与约束硬化)}:算法治理行动改变了平台决策的约束条件集合。设平台内容分发策略空间为$\mathcal{P}$,监管前平台面临基于用户点击率的软约束,监管后《治理清单》第7条将多样性要求转化为硬约束。定义$H(p)$为衡量话题分布多样性的Shannon熵,则监管导致的多样性下限$\delta$发生位移:
\begin{equation}
\text{s.t. } H(p^{\text{post}}) \ge \delta_{\text{post}} > \delta_{\text{pre}}
\end{equation}
该约束强化导致平台可行域收缩,剔除了虽然商业价值最大化但熵值低于$\delta_{\text{post}}$的极点解。

\textit{命题1.2(分布的优劣关系与头部去中心化)}:定义$N$个话题的注意力分布向量为$p = (p_1, p_2, \dots, p_N)$,并按降序排列$p_1 \ge p_2 \ge \dots \ge p_N$。监管措施通过内容去重与打散干预强制削减前$k$个头部话题的流量配额,并将其再分配至长尾话题。数学上,这等价于监管后的分布$p^{\text{post}}$被监管前的分布$p^{\text{pre}}$所优势,记作$p^{\text{pre}} \succ p^{\text{post}}$。该关系满足洛伦兹优势条件:
\begin{equation}
\begin{cases}
\displaystyle \sum_{i=1}^{m} p_i^{\text{pre}} \ge \sum_{i=1}^{m} p_i^{\text{post}}, & \forall m \in \{1, \dots, N-1\} \\[1em]
\displaystyle \sum_{i=1}^{N} p_i^{\text{pre}} = \sum_{i=1}^{N} p_i^{\text{post}} = 1
\end{cases}
\end{equation}
此不等式组表明,在任意截断点$m$,监管前的累积注意力集中度均不低于监管后。

\textit{定理1(Shannon熵的舒尔凹性)}:当注意力分布从高度集中转向相对均匀时,Shannon熵单调递增。Shannon熵定义为:
\begin{equation}
H(p) = -\sum_{i=1}^{N} p_i \ln p_i = \sum_{i=1}^{N} \phi(p_i)
\end{equation}
其中$\phi(x) = -x \ln x$。计算$\phi(x)$的二阶导数:$\phi''(x) = -1/x < 0$,在定义域$x \in (0, 1]$上恒成立,故$\phi(x)$为严格凹函数。根据优序理论,若$p^{\text{pre}} \succ p^{\text{post}}$,则必然蕴含$H(p^{\text{post}}) > H(p^{\text{pre}})$。

\textbf{2. 理论机制解释}

\textit{机制1.1:目标函数重构——从单目标优化到约束优化}。监管介入前,平台的决策模型可抽象为无约束的流量最大化问题。由于头部话题通常具有更高的商业变现效率,平台倾向于产生极度偏斜的分布。监管实质上引入了拉格朗日乘数$\lambda$,新目标函数为:
\begin{equation}
\mathcal{L}(p, \lambda) = \sum_{i=1}^{N} v_i p_i + \lambda \left( H(p) - \delta_{\text{post}} \right)
\end{equation}
其中影子价格$\lambda > 0$代表平台为满足合规要求所必须放弃的边际商业收益。\citet{OestreicherSinger2012}的研究表明,推荐系统能够显著改变产品需求的分布结构,当平台调整算法以提升长尾产品的可见性时,需求分布会从高度集中向更均匀的方向移动。这一机制表明,多样性提升并非市场自发演化的结果,而是通过制度设计强制改变了帕累托最优解的对偶性质。

\textit{机制1.2:资源编排逻辑——从马太效应到累进调节}。推荐算法的正反馈机制导致了注意力分配的马太效应。《治理清单》中的去重与打散构成了双向调节机制,其作用机理类似于税收制度中的累进税:对高流量话题征收``关注度税'',并转移支付给长尾话题。这种非线性的干预手段切断了``高曝光--高点击--更高曝光''的循环链路,强制系统熵值向最大熵状态漂移。\citet{Berman2018}通过理论模型发现,当推荐系统提升内容多样性时,虽然短期内可能降低用户参与度,但长期而言能够提升用户满足度和平台健康度。

\textit{机制1.3:长尾激活——从被动沉默到主动参与}。在监管前的自然演化状态下,大量长尾话题由于缺乏初始曝光而陷入沉默状态,即使其潜在受众规模不小,也难以突破注意力门槛。算法治理通过为长尾内容预留基础曝光配额,降低了话题进入榜单的边际成本。这种保底机制激活了原本沉默的长尾话题,使其获得接触目标受众的机会,进而可能形成自我维持的关注度增长。

\textbf{3. 假说陈述}

基于上述数学推导与理论机制,提出如下假说:

\textbf{H1(注意力去中心化)}:算法治理打破了热搜榜单的赢者通吃格局,注意力从少数头部话题向广泛的长尾话题再分配。

H1a(熵值上升):$\mathbb{E}[H^{\text{post}}] > \mathbb{E}[H^{\text{pre}}]$

H1b(集中度下降):$\mathbb{E}[\text{HHI}^{\text{post}}] < \mathbb{E}[\text{HHI}^{\text{pre}}]$,其中HHI定义为$\text{HHI} = \sum_{i=1}^{N} p_i^2$

\subsubsection{假说H2:内容质量分化效应}

\textbf{1. 数学推导}

\textit{命题2.1(合规成本函数的超模性结构)}:设平台内容审核的成本函数为$C(Q, R)$,其中$Q \in \mathbb{R}_+$为内容质量,$R \in \mathbb{R}_+$为内容的风险等级。社会类内容由于具有较强舆论属性和社会动员能力,其风险等级$R_S$显著高于娱乐类内容$R_E$,即$R_S > R_E$。该成本函数满足以下性质:

其一,质量的边际成本递增:$\frac{\partial C}{\partial Q} > 0, \quad \frac{\partial^2 C}{\partial Q^2} > 0$

其二,风险的成本放大效应:$\frac{\partial C}{\partial R} > 0$

其三,质量与风险的互补性:$\frac{\partial^2 C}{\partial Q \partial R} > 0$

即在高风险情境下提升单位质量的边际成本显著高于低风险情境。形式化而言,采用广义Cobb--Douglas形式与指数风险项结合的成本函数:
\begin{equation}
C(Q, R) = \alpha \cdot Q^{\beta} \cdot e^{\gamma R}, \quad \beta > 1,\ \gamma > 0
\end{equation}
该设定保证了上述三条性质成立,并刻画了高风险内容审核成本的指数级敏感性。

\textit{命题2.2(差异化约束的优化模型)}:平台面临在满足监管约束前提下最小化运营成本的决策问题。算法治理行动对不同风险类别施加非对称的质量下限约束$\underline{Q}(R)$:
\begin{equation}
\begin{aligned}
\min_{Q_S, Q_E} \quad & C(Q_S, R_S) + C(Q_E, R_E) \\
\text{s.t.} \quad & Q_S \ge \underline{Q}_S^{\text{post}}, \\
& Q_E \ge \underline{Q}_E^{\text{post}}.
\end{aligned}
\end{equation}
监管对高风险社会类内容的合规门槛提升更为显著:
\begin{equation}
\underline{Q}_S^{\text{post}} \gg \underline{Q}_S^{\text{pre}}, \quad
\underline{Q}_E^{\text{post}} \approx \underline{Q}_E^{\text{pre}}
\end{equation}

\textit{定理2(角点解与质量分化)}:在上述设定下,考虑凸成本函数与线性不等式约束的优化问题。根据KKT条件,最优解将在约束恰好绑定或内部点之间选择。对于社会类,由于$\underline{Q}_S^{\text{post}}$设定了较高的质量下限,且在高风险水平$R_S$下,边际成本随$Q$增长极快,平台缺乏过度合规的激励,因此最优解倾向于在约束边界取得角点解:$Q_S^* = \underline{Q}_S^{\text{post}} > Q_S^{\text{pre}}$。对于娱乐类,$R_E$较小使得边际成本曲线更为平缓,且$\underline{Q}_E^{\text{post}}$与监管前相比并未发生显著位移,平台在既有质量水平上即可满足最低合规要求:$Q_E^* \approx Q_E^{\text{pre}}$。

\textbf{2. 理论机制解释}

\textit{机制2.1:基于风险的监管逻辑}。《算法专项治理清单指引》在异常账号监测与榜单操纵治理以及优化内容生态等模块中,对涉及公共事件、舆情风险和违法谣言的信息分发提出了更高要求。这体现了基于风险的监管逻辑:监管者根据潜在负外部性的大小分配治理资源与规范强度\citep{Song2018}。具有较强公共属性和社会动员能力的社会类话题被视为高风险类别,一旦出现虚假信息或价值偏离,可能引发系统性舆情风险,因此被纳入更严格的治理序列,面对更高的信息来源、事实核验和价值导向标准。相对而言,娱乐类话题主要关联个体消费与情绪调节,其外部性更局限于低俗、侵权等局部问题,监管多采取守住底线的防御性姿态。

\textit{机制2.2:资源依赖与挤出效应}。平台用于内容治理的预算、专业人力和算力等资源在短期内具有刚性约束。\citet{Huber2013}指出,当治理成本增量超过治理价值增量时,平台会调整其差异化策略以维持生态系统的整体效益。算法治理行动显著提高了社会类内容的边际合规成本后,高风险领域便成为组织必须优先满足的关键依赖对象。为了确保此类内容不触碰监管红线,平台不得不在资源配置上进行再平衡,将原本可用于优化长尾内容或娱乐内容体验的部分资源,转而集中投入到社会类话题的审核、风控与事后处置之中。

\textit{机制2.3:信号显示与合法性获取}。在信息不对称的平台治理情境下,内容质量的提升构成平台向监管者发送合规信号的重要途径。\citet{Song2018}的研究表明,平台通过差异化治理策略来平衡不同利益相关者的诉求。对于高风险的社会类内容,平台倾向于采取更为显著的质量提升措施,如增加官方媒体占比、减少标题党内容,以向监管者展示其积极履行社会责任的姿态。

\textbf{3. 假说陈述}

基于上述数学推导与理论机制,提出如下假说:

\textbf{H2(内容供给质量分化)}:算法治理导致不同风险类别内容的质量呈现分化趋势,高风险内容质量显著提升,低风险内容质量基本稳定。

H2a(社会类质量提升):社会类话题的信息差指数(衡量标题党程度)在监管后显著下降

H2b(官媒相对优势):社会类内部,官方媒体相对非官方媒体的信息差净下降(DID估计量显著为负)

H2c(娱乐类稳定):娱乐类话题的质量指标在监管前后无显著变化

\subsubsection{假说H3:注意力再分配效应(水床效应)}

\textbf{1. 数学推导}

\textit{命题3.1(注意力预算约束与影子价格)}:定义平台内容生态的总注意力预算为$\bar{A}$。在短期内,受限于用户认知负荷与日活跃用户规模的相对稳定,总注意力供给缺乏弹性,可视为外生常数。平台面临的约束优化问题可表示为:
\begin{equation}
\max_{\{A_i\}} \ \Pi = \sum_{i \in \{S,E,O\}} \big[ R_i(A_i) - C_i(A_i,\xi_i) \big] + \mu\, D(A_S,A_E,A_O)
\end{equation}
\begin{equation}
\text{s.t.} \quad \sum_{i} A_i \le \bar{A}, \quad A_i \ge 0
\end{equation}
其中,$i$代表内容类别($S=$社会,$E=$娱乐,$O=$其他);$R_i(A_i)$为凹收益函数($R_i'>0,\ R_i''<0$);$C_i(A_i,\xi_i)$为凸成本函数,$\xi_i$为外生监管强度参数;$D(\cdot)$为多样性激励项,$\mu \ge 0$为多样性权重。

构建拉格朗日函数:
\begin{equation}
\mathcal{L} = \sum_{i} \big[ R_i(A_i) - C_i(A_i,\xi_i) \big] + \mu D(A_S,A_E,A_O) + \lambda \Big(\bar{A} - \textstyle\sum_{i} A_i\Big)
\end{equation}
其中$\lambda \ge 0$为注意力资源的影子价格,表示在最优点附近增加1单位注意力预算所带来的边际收益。

\textit{命题3.2(不对称成本冲击下的替代效应)}:对每一类内容$i$,若最优解处$A_i^*>0$,一阶条件(FOC)为:
\begin{equation}
\frac{\partial R_i}{\partial A_i} - \frac{\partial C_i}{\partial A_i} + \mu \frac{\partial D}{\partial A_i} = \lambda, \quad i \in \{S,E,O\}
\end{equation}
这表明在最优点上,各类内容的``边际收益$-$边际成本$+$多样性边际贡献''在注意力空间内应当被拉平到同一水平$\lambda$。

现在考察监管冲击导致社会类监管强度参数$\xi_S$上升的情形。根据链式法则与比较静态分析,当$\xi_S$上升时,社会类内容的边际合规成本上升。要维持FOC成立,平台可以通过调整$A_S$与其他类别的$A_E, A_O$使得新的均衡满足条件。在$R_S$凹、$C_S$凸的假定下,为抵消边际成本的上升,最直接的调整方式是减少$A_S$,从而提高$\frac{\partial R_S}{\partial A_S}$并降低$\frac{\partial C_S}{\partial A_S}$。在注意力预算约束紧约束的情况下:
\begin{equation}
\Delta A_S < 0 \quad \Rightarrow \quad \Delta A_E + \Delta A_O > 0
\end{equation}

在娱乐类成本函数对监管强度不敏感($\partial C_E / \partial \xi_S \approx 0$),且其边际收益率通常高于其他类($R'_E > R'_O$)的假定下,边际净收益均等原则将驱动释放出的注意力资源优先流向娱乐板块。由此导出``水床效应''的比较静态结论:
\begin{equation}
\frac{\partial A_E^*}{\partial \xi_S} > 0,\quad \frac{\partial A_S^*}{\partial \xi_S} < 0
\end{equation}
即社会类注意力在监管压力上升时收缩,释放出的注意力在总量约束下向娱乐类转移。

\textbf{2. 理论机制解释}

\textit{理论解释3.1:存量注意力博弈下的挤出与回填}。注意力经济的核心特征在于认知资源的稀缺性。在移动互联网渗透率趋于饱和的背景下,用户总时长与活跃规模在短期内呈现强刚性,总注意力$\bar{A}$可以近似视为固定存量\citep{OestreicherSinger2012}。在此情境下,监管对社会类内容的强化治理,相当于在该类别上抬高风险权重与合规成本,使其边际净收益相对其他类别下降,从而促使平台在约束优化过程中压缩社会类的注意力权重。然而,被削减的社会类注意力并不会简单消失,而是在总注意力约束下寻找新的配置路径。根据平台``水床效应''的分析视角\citep{Zhu2019},在不同内容类别之间存在类似``连通容器''的流量转移:当高风险类别因监管而被压制时,流量会通过算法分发机制向成本更低、约束更松的内容板块上升。

\textit{理论解释3.2:边际净收益的动态均衡}。从平台算法的视角看,推荐系统本质上是在多类别内容之间不断进行``边际净收益''比较与再平衡的过程\citep{He2022}。监管前,社会类与娱乐类在收益—风险维度上处于某种相对稳定的权衡状态;监管后,社会类内容的风险溢价大幅上升,其风险调整后的回报显著下降。理性的算法策略会减少对高风险内容的推荐权重,将注意力向风险较低、收益相对稳定的娱乐类内容倾斜\citep{Kleinberg2021}。

\textit{理论解释3.3:合规治理的意外后果}。从治理目标看,监管者的初衷在于压缩有害信息的传播空间、提升整体内容质量;但在复杂适应系统中,强约束往往会通过注意力再分配机制产生二阶效应。对高风险社会类话题施加更高合规门槛与更强审慎性要求,一方面确实减少了明显违规与失真的信息,另一方面也可能在算法和审核实践中``顺带''压缩了一部分合规但敏感度较高的公共议题,从而在相对意义上扩大了娱乐性内容在整体注意力中的占比。这体现出典型的``合规政策的意外后果''特征\citep{Gorwa2020}。

\textbf{3. 假说陈述}

综合上述数理推导与理论机制,可以提出如下假说:

\textbf{H3(注意力再分配/水床效应)}:在注意力零和约束下,对社会类的监管强化将引发注意力向娱乐类的系统性转移。

H3a(社会类份额下降):社会类话题在热搜榜中的份额在监管后显著下降

H3b(娱乐类份额上升):娱乐类话题在热搜榜中的份额在监管后显著上升

H3c(总量守恒):周度总热度在监管前后无显著变化,验证注意力零和约束

\subsubsection{假说H4:注意力密度分化效应}

\textbf{1. 数学推导}

\textit{命题4.1(类内注意力密度的定义)}:定义类别$i$在时期$t$的注意力密度(attention density)为单个话题的平均注意力强度:
\begin{equation}
\rho_i(t) = \frac{A_i(t)}{N_i(t)}
\end{equation}
其中$A_i(t)$为类别$i$获得的总注意力资源,$N_i(t)$为该类别在榜话题数量。注意力密度$\rho_i(t)$反映了单个话题在类内的``平均曝光强度''或``单位话题吸引力'',是衡量内容竞争激烈程度的逆指标:密度越高,说明单个话题分摊到的注意力越充裕;密度越低,则意味着内容供给相对过剩,单个话题面临更激烈的类内竞争。

\textit{命题4.2(社会类的``少而精''调整机制)}:根据H3的推导,算法治理实施后社会类总注意力收缩,即$A_S^{\text{post}} < A_S^{\text{pre}}$。与此同时,监管对社会类内容施加了更严格的质量门槛$Q_S^{\text{post}} \gg Q_S^{\text{pre}}$(参见H2)。在质量约束紧约束的情况下,大量低质量或合规风险较高的候选话题将被直接排除在榜外,导致在榜话题数量$N_S^{\text{post}}$显著下降。

假定总注意力$A_S$的收缩幅度小于在榜话题数量$N_S$的收缩幅度,即存在常数$\alpha, \beta \in (0,1)$满足:
\begin{equation}
A_S^{\text{post}} = (1-\alpha) A_S^{\text{pre}}, \quad N_S^{\text{post}} = (1-\beta) N_S^{\text{pre}}, \quad \beta > \alpha
\end{equation}
在此假定下,社会类的注意力密度变化为:
\begin{equation}
\rho_S^{\text{post}} = \frac{A_S^{\text{post}}}{N_S^{\text{post}}} = \frac{(1-\alpha) A_S^{\text{pre}}}{(1-\beta) N_S^{\text{pre}}} = \frac{1-\alpha}{1-\beta} \cdot \rho_S^{\text{pre}}
\end{equation}
由于$\beta > \alpha$,有$(1-\alpha)/(1-\beta) > 1$,从而$\rho_S^{\text{post}} > \rho_S^{\text{pre}}$。

解释:社会类总注意力虽然下降,但``准入门槛''提高导致在榜话题数量下降幅度更大,分母的收缩速度超过分子,结果是单个社会类话题的平均注意力密度反而上升。这种``少而精''的调整机制,使得通过质量筛选留存下来的社会类话题享受到更高的单位曝光强度。

\textit{命题4.3(娱乐类的``多且长''效应)}:与社会类相反,娱乐类在算法治理后总注意力上升($A_E^{\text{post}} > A_E^{\text{pre}}$,参见H3b),但由于其合规成本增长有限、准入门槛相对宽松,在榜话题数量也会同步增加。娱乐类在总量、份额、密度三个维度均呈上升态势,形成``多且长''的调整格局。

\textbf{2. 理论机制解释}

\textit{理论解释4.1:质量门槛的筛选机制与``准入竞争''}。算法治理的一个核心特征在于通过提高质量门槛实现``事前筛选''(ex-ante filtering),而非单纯依赖``事后惩罚''(ex-post removal)。对于社会类话题,监管清单中的多项条款均要求平台在推荐前对内容进行更严格的审核与质量评估\citep{He2022}。这一机制将大量``边缘话题''——即质量勉强达标但合规风险不确定的内容——排除在榜外,从而压缩了社会类的在榜规模。与此同时,总注意力的收缩主要通过``降权''而非``完全下架''实现:平台仍会保留部分高质量社会类话题的推荐,只是将其在整体注意力分配中的权重下调。结果是,通过筛选留存下来的话题数量减少,但它们占据的注意力份额相对稳定,单位话题的曝光强度因此上升。

\textit{理论解释4.2:供给涌入与内容拥塞的负外部性}。娱乐类内容在算法治理后享受到更低的合规成本与更宽松的准入环境,这会引发供给端的``涌入效应'':大量创作者与内容聚合方察觉到娱乐类话题的推荐权重上升,纷纷增加娱乐类内容的生产与投放,以期获取流量红利。

\textbf{3. 假说陈述}

综合上述数理推导与理论机制,提出如下假说:

\textbf{H4(注意力密度分化)}:在注意力再分配过程中,不同类型内容的注意力密度呈现差异化演化路径。

H4a(社会类密度稳定或上升):社会类单条话题的平均在榜时长在监管后保持稳定或上升,体现``少而精''调整

H4b(娱乐类密度上升):娱乐类单条话题的平均在榜时长、总在榜时长、时间份额在监管后显著上升,体现``多且长''调整

综上所述,本文构建了一个整合平台治理理论、注意力经济理论和双边网络理论的分析框架,提出了四大核心假说:H1预测注意力去中心化,H2预测内容质量分化,H3预测注意力再分配(水床效应),H4预测注意力密度分化。
