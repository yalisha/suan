\section{异质性分析}

为进一步探讨算法治理效应的异质性,本文从季节周期、时间段、话题热度三个维度展开分析,结果见表\ref{tab:heterogeneity}。

\begin{table}[htbp]
\centering
\caption{异质性检验结果(ITS回归)}
\label{tab:heterogeneity}
\small
\begin{tabular}{llcccc}
\toprule
{\tabkai 异质性维度} & {\tabkai 分组} & {\tabkai 即时效应} & $p${\tabkai 值} & {\tabkai 趋势效应} & $p${\tabkai 值} \\
\midrule
\multicolumn{6}{l}{\textit{\tabkai Panel A:季度异质性(信息差指数)}} \\
& Q1 & $-$0.0129$^{*}$ & 0.050 & 0.0001 & 0.587 \\
& Q2 & $-$0.0245$^{***}$ & $<$0.001 & $-$0.0001 & 0.491 \\
& Q3 & $-$0.0021 & 0.641 & $-$0.0000 & 0.566 \\
& Q4 & $-$0.0087 & 0.160 & $-$0.0001 & 0.657 \\
\midrule
\multicolumn{6}{l}{\textit{\tabkai Panel B:时间段异质性(情绪唤起指数)}} \\
& {\tabkai 工作日} & 0.0013 & 0.725 & $-$0.0000 & 0.915 \\
& {\tabkai 周末} & $-$0.0129$^{*}$ & 0.017 & $-$0.0000 & 0.711 \\
\midrule
\multicolumn{6}{l}{\textit{\tabkai Panel C:话题热度异质性(信息差指数)}} \\
& {\tabkai 高热度} & $-$0.0074 & 0.196 & $-$0.0000 & 0.461 \\
& {\tabkai 中热度} & 0.0056 & 0.290 & 0.0001$^{**}$ & 0.006 \\
& {\tabkai 低热度} & 0.0055 & 0.304 & 0.0000 & 0.167 \\
\bottomrule
\end{tabular}
\begin{tablenotes}
\small
\item {\tabkai 注:}$^{***}p<0.001$,$^{**}p<0.01$,$^{*}p<0.05${\tabkai ;样本为社会类话题;所有模型均控制季度固定效应与节假日效应;热度分组基于三分位数划分。}
\end{tablenotes}
\end{table}

\subsection{季度异质性}

信息差指数的政策效应呈现显著的季度差异。Q2(4-6月)效应最为强烈,即时下降2.45个百分点($p<0.001$);Q1(1-3月)次之,即时下降1.29个百分点($p=0.050$);Q3与Q4效应则不显著。这一季节性模式可能与监管资源配置周期有关:Q1-Q2通常是监管部门年度考核与专项行动密集期,平台合规压力较大;下半年监管力度相对趋缓,治理效应有所减弱。

\subsection{时间段异质性}

情绪唤起指数在工作日与周末呈现差异化响应。周末情绪唤起指数即时下降1.29个百分点($p=0.017$),而工作日效应不显著($p=0.725$)。这一发现揭示了用户行为与算法推荐的交互模式:周末用户在线时长较长、信息消费更为集中,平台在此时段更倾向于推送高情绪价值内容以提升用户黏性;政策实施后,周末的情绪激励策略受到更大约束,效应更为显著。

\subsection{话题热度异质性}

按热度三分位划分样本后,高热度话题呈现即时下降趋势但不显著($p=0.196$);中热度话题在趋势效应上显著为正($p=0.006$),表明政策后中热度话题的信息差呈持续上升态势;低热度话题两项效应均不显著。这一非对称格局暗示,算法治理主要约束了头部高热度话题的信息诱导策略,而中热度话题作为``腰部内容''在监管盲区中获得了更大的策略空间。

