\section{样本选择与数据来源}

\subsection{数据来源与样本筛选}

本研究使用新浪微博热搜榜的历史数据作为核心数据源。微博作为中国最大的社交媒体平台之一,其热搜榜单通过算法实时聚合用户浏览、搜索、讨论等行为数据,动态生成并展示当前最受关注的50个话题,是观察算法治理效应的理想场景。数据采集的时间窗口为2024年2月1日至2025年10月1日,跨度共计608天,完整覆盖了2024年11月12日算法治理政策实施的前后阶段,其中政策前观测期为253天,政策后观测期为355天。

原始数据包含话题名称、热度值(微博官方计算的综合指标,整合了阅读量、讨论量、搜索量等多维度数据)、上榜时长、话题主持人分类(微博平台预设的81个细分类别)、时间戳(精确到分钟级别)等关键维度。为确保数据质量,本文对原始数据实施了三步筛选:第一,剔除时间戳缺失或异常的记录;第二,删除空话题或系统占位符;第三,排除平台未明确分类的``其他''类别话题(该类别含义模糊,无法纳入类别对比分析)。经过筛选,最终样本包含253,227条观测值,对应约215,083个独立话题,覆盖608个日样本。

\subsection{话题分类方法}

本研究直接沿用微博平台预设的主持人分类字段作为话题分类的基础。选择平台原始分类的理由有三:第一,减少测量误差,平台分类基于大规模用户行为数据与算法模型训练,相较于研究者的二次编码具有更高的稳定性;第二,贴近平台决策逻辑,算法治理政策的实施主体是平台本身,使用平台原始分类可最大程度还原算法调整的真实作用路径;第三,提升研究可复制性,平台分类数据可直接获取且公开透明。

虽然原始数据包含81个细分类别,但为聚焦理论假设中的核心对比——``高风险社会类内容''与``低风险娱乐类内容''之间的差异化效应——本文选取了占比排名前十的类别作为主分析对象。这十大类别合计覆盖约86\%的样本量,具体分布见表\ref{tab:category}。

\begin{table}[htbp]
\centering
\caption{Top 10话题类别分布}
\label{tab:category}
\begin{tabular}{lcccl}
\toprule
{\tabkai 类别} & {\tabkai 占比} & {\tabkai 累计占比} & {\tabkai 风险等级} & {\tabkai 核心分析组} \\
\midrule
{\tabkai 社会} & 34.82\% & 34.82\% & {\tabkai 高风险} & {\tabkai 社会类} \\
{\tabkai 明星} & 11.86\% & 46.68\% & {\tabkai 低风险} & {\tabkai 娱乐类(核心)} \\
{\tabkai 体育} & 10.43\% & 57.11\% & {\tabkai 中风险} & --- \\
{\tabkai 时事} & 6.52\% & 63.63\% & {\tabkai 中风险} & --- \\
{\tabkai 游戏} & 4.82\% & 68.45\% & {\tabkai 低风险} & --- \\
{\tabkai 明星-内地} & 4.61\% & 73.06\% & {\tabkai 低风险} & {\tabkai 娱乐类} \\
{\tabkai 综艺} & 4.30\% & 77.36\% & {\tabkai 低风险} & --- \\
{\tabkai 电视剧} & 3.17\% & 80.53\% & {\tabkai 低风险} & --- \\
{\tabkai 搞笑} & 2.69\% & 83.22\% & {\tabkai 低风险} & --- \\
{\tabkai 财经} & 2.46\% & 85.68\% & {\tabkai 中风险} & --- \\
\bottomrule
\end{tabular}
\end{table}

基于理论假设中关于``监管强度的非对称性'',本文将上述类别聚合为两个核心对比组:社会类(占比34.82\%),主要涵盖公共事件、民生议题、社会现象等具有较强舆论属性的内容,根据《算法专项治理清单指引》第11条和第23条,此类内容被纳入平台重点监测范围,面临更高的信息真实性、来源可溯性要求,构成典型的``高风险、高合规成本''类别;娱乐类(包括``明星''与``明星-内地'',合计占比16.47\%),主要涵盖明星动态、娱乐八卦、粉丝互动等内容,其特征是用户参与度高但舆论风险相对可控,监管多采取``守住底线''的防御性姿态,构成典型的``低风险、低合规成本''类别。选择这两个类别作为核心对比组,是因为二者在监管强度、内容属性、用户群体等维度上呈现出最鲜明的两极分化特征,这种``极端对比''策略有助于清晰识别算法治理的差异化效应。

\subsection{变量定义与测量}

本研究构建了涵盖多样性、质量、注意力配置、注意力密度四个维度的指标体系,以全面检验算法治理对平台生态系统的重构效应。表\ref{tab:variables}汇总了所有核心变量的定义、测量方法、文献来源及对应假说。

\subsubsection{多样性指标(H1)}

Shannon熵是信息论中衡量分布不确定性的经典指标,广泛应用于推荐系统多样性的评估\citep{OestreicherSinger2012}。该指标值越大,表示注意力分布越均匀,头部垄断程度越低。定义为:
\begin{equation}
H(p) = -\sum_{i=1}^{N} p_i \ln p_i
\end{equation}
其中,$p_i$为第$i$个话题获得的注意力份额(以在榜时长占比衡量),$N$为观测日内上榜话题总数。当所有话题获得等额注意力时($p_i = 1/N$),Shannon熵达到最大值$\ln N$;当注意力完全集中于单一话题时($p_1 = 1, p_{i \neq 1} = 0$),熵值降至0。

HHI指数源于产业组织理论,用于衡量市场集中度\citep{Berman2018}。在注意力经济情境下,HHI反映了少数头部话题对总注意力的垄断程度。定义为:
\begin{equation}
\text{HHI} = \sum_{i=1}^{N} p_i^2
\end{equation}
HHI的取值范围为$[1/N, 1]$。当注意力完全均匀分布时,$\text{HHI} = 1/N$,集中度最低;当注意力完全集中于单一话题时,$\text{HHI} = 1$,集中度最高。

\subsubsection{质量指标(H2)}

信息差指数基于\citet{Loewenstein1994}的信息差理论(information gap theory),该理论指出当个体意识到知识缺口时会产生强烈好奇心,驱使其采取行动填补缺口。标题党(clickbait)正是利用这一心理机制,通过制造标题与内容之间的语义悬念来操纵用户的注意力分配\citep{Golman2018}。本研究采用基于关键词词频统计的方法构建信息差指数,该方法在社交媒体内容分析中得到广泛应用\citep{Chakraborty2016}。

具体而言,本文识别话题标题中制造好奇缺口的三类语言特征:疑问词(为何、为什么、怎么、什么、原因、真相)、悬念词(曝、竟、暗示、去向、结局、神反转、揭秘、内幕、背后)、疑问标点(?、?)。只要标题中出现任一上述特征,即判定为存在信息差(赋值为1),否则为0。该二元测度直接捕捉了标题是否利用``知识缺口''策略吸引注意力,值越高表示标题党程度越严重,内容质量越低。

为检验H2b中提出的``官媒相对优势''假说,本文对话题的信源属性进行了分类。根据\citet{He2022}对中国舆论生态的研究,官方媒体在内容规范性和信息可信度方面具有显著优势,构成平台治理中的``质量基准''。本研究基于人工标注数据(manual\_host\_labels.csv)对话题主持人进行官方/非官方媒体分类,该标注结合了启发式规则和人工校验。官方媒体定义为:党媒(如人民日报、新华社)、政务账号(如公安、卫健委)及传统主流媒体(如央视新闻、澎湃新闻)。官方媒体虚拟变量赋值为1,其他账号赋值为0。

\subsubsection{注意力配置指标(H3)}

注意力份额衡量特定类别内容在热搜榜总注意力资源中的占比,反映了平台算法在不同内容类型间的资源配置偏好。\citet{Hosanagar2014}在研究用户注意力的稀缺性时指出,注意力资源具有零和性质,对某一类型的注意力分配必然减少其他类型的份额。本研究定义注意力份额为:
\begin{equation}
\text{Share}_{it} = \frac{\sum_{j \in i} \text{Duration}_{jt}}{\sum_{k} \text{Duration}_{kt}}
\end{equation}
其中,$i$代表内容类别(社会类或娱乐类),$t$代表时间(日度),$\text{Duration}_{jt}$为话题$j$在$t$日的在榜时长,分子为类别$i$内所有话题的总在榜时长,分母为当日所有话题的总在榜时长。

周度总热度用于验证H3c中提出的``注意力零和约束''。如果用户的总注意力预算在短期内保持稳定,则算法治理不应显著改变平台的总注意力供给量,而仅改变其在不同类型间的分配结构。定义为:
\begin{equation}
\text{TotalHeat}_w = \sum_{t \in w} \sum_j \text{Heat}_{jt}
\end{equation}
其中,$w$代表周(7天为一个观测窗口),$\text{Heat}_{jt}$为话题$j$在$t$日的热度值。

\subsubsection{注意力密度指标(H4)}

注意力密度衡量单位内容获得注意力的强度,揭示了在总量重新分配的同时,单条话题获得注意力的深度变化\citep{Leskovec2009}。本研究定义类内注意力密度为:
\begin{equation}
\text{Density}_{it} = \frac{\sum_{j \in i} \text{Duration}_{jt}}{\text{Count}_{it}}
\end{equation}
其中,$\text{Count}_{it}$为类别$i$在$t$日的上榜话题数量,分子为类别总在榜时长,分母为话题数量。该指标分解了总注意力变化的两个维度:广度(话题数量)与深度(单条话题强度)。

\begin{table}[htbp]
\centering
\caption{变量定义与测量}
\label{tab:variables}
\small
\begin{tabular}{p{3cm}p{2cm}p{8cm}}
\toprule
{\tabkai 变量名称} & {\tabkai 变量符号} & {\tabkai 变量定义} \\
\midrule
\multicolumn{3}{l}{\textit{\tabkai 多样性指标}} \\
{\tabkai Shannon熵} & $\var{H}$ & $H(p) = -\sum_{i=1}^{N} p_i \ln p_i$,$p_i${\tabkai 为第}$i${\tabkai 个话题的注意力份额} \\
{\tabkai HHI指数} & $\var{HHI}$ & $\text{HHI} = \sum_{i=1}^{N} p_i^2$,{\tabkai 衡量注意力集中度} \\
\midrule
\multicolumn{3}{l}{\textit{\tabkai 质量指标}} \\
{\tabkai 信息差指数} & $\var{InfoGap}$ & {\tabkai 二元变量,标题含疑问词/悬念词/疑问标点=1,否则=0} \\
{\tabkai 官方媒体} & $\var{Official}$ & {\tabkai 党媒/政务/主流媒体=1,其他=0(基于人工标注)} \\
\midrule
\multicolumn{3}{l}{\textit{\tabkai 注意力配置指标}} \\
{\tabkai 注意力份额} & $\var{Share}$ & {\tabkai 类别}$i${\tabkai 在}$t${\tabkai 日的在榜时长占全部话题在榜时长的比例} \\
{\tabkai 周度总热度} & $\var{TotalHeat}$ & {\tabkai 一周内所有话题热度值的加总} \\
\midrule
\multicolumn{3}{l}{\textit{\tabkai 注意力密度指标}} \\
{\tabkai 单条话题平均时长} & $\var{Density}$ & {\tabkai 类别总在榜时长除以该类别上榜话题数量} \\
\bottomrule
\end{tabular}
\end{table}

\subsection{计量模型设定}

本研究采用基于OLS的断点回归时间序列分析与双重差分(DID)相结合的识别策略,以充分利用政策冲击的外生性和数据的面板结构。OLS时间序列模型用于检验H1(注意力去中心化)、H3(注意力再分配)和H4(注意力密度分化),通过对比政策实施前后整体层面的趋势变化识别政策的断点效应;对于H2(内容质量分化),本文先采用分组OLS回归检验社会类与娱乐类的整体质量变化,再采用DID模型作为稳健性检验,通过对比社会类内部官方媒体与非官方媒体的相对变化,进一步验证质量分化机制。

\subsubsection{模型1:注意力去中心化检验(H1)}

基于日度时间序列数据,本文采用OLS回归模型检验政策实施后Shannon熵和HHI的断点变化:
\begin{equation}
Y_t = \beta_0 + \beta_1 \text{Post}_t + \beta_2 \text{Trend}_t + \beta_3 (\text{Post}_t \times \text{Trend}_t) + \varepsilon_t
\end{equation}
其中,$Y_t$为$t$日的Shannon熵或HHI,$\text{Post}_t$为政策实施虚拟变量(2024年11月12日及之后=1,之前=0),$\text{Trend}_t$为线性时间趋势(从1开始递增的日序列),$\text{Post}_t \times \text{Trend}_t$为政策后趋势交互项。$\beta_1$捕捉政策的即时效应(level change),即政策实施当日Shannon熵或HHI的跳跃性变化;$\beta_3$捕捉政策的趋势效应(slope change),即政策后Shannon熵或HHI的增长速度(或下降速度)相对政策前的变化。标准误采用Newey-West方法校正以处理潜在的序列相关性。

\subsubsection{模型2:内容质量分化检验(H2)}

本文采用两阶段嵌套检验策略。第一阶段(主效应检验)通过分组OLS回归对比社会类与娱乐类在信息差指数上的政策前后变化:
\begin{equation}
\text{InfoGap}_{it} = \beta_0 + \beta_1 \text{Post}_t + \beta_2 \text{Trend}_t + \beta_3 (\text{Post}_t \times \text{Trend}_t) + \varepsilon_{it}
\end{equation}
分别对社会类和娱乐类样本运行该模型,检验各自的质量变化趋势。

第二阶段(稳健性检验)在社会类内部构建DID模型,以官媒话题为对照组、非官媒话题为处理组:
\begin{equation}
\text{InfoGap}_{it} = \alpha + \delta (\text{Post}_t \times \text{Official}_i) + \gamma_i + \lambda_t + \varepsilon_{it}
\end{equation}
其中,$\text{Official}_i$为官方媒体虚拟变量(基于人工标注数据),$\gamma_i$为话题固定效应,$\lambda_t$为时间固定效应。$\delta$为DID估计量,捕捉政策实施后非官方媒体相对官方媒体的信息差净变化。

\subsubsection{模型3:注意力再分配检验(H3)}

与模型1结构相同,但因变量改为特定类别的注意力份额。本文分别对社会类和娱乐类运行OLS回归:
\begin{equation}
\text{Share}_{it} = \beta_0 + \beta_1 \text{Post}_t + \beta_2 \text{Trend}_t + \beta_3 (\text{Post}_t \times \text{Trend}_t) + \varepsilon_t
\end{equation}
根据H3a和H3b,预期社会类的$\beta_1 < 0$(份额即时下降)且$\beta_3$可能为负(持续下降趋势),娱乐类的$\beta_1 > 0$(份额即时上升)且$\beta_3$可能为正。

\subsubsection{模型4:注意力密度分化检验(H4)}

对社会类和娱乐类分别运行OLS回归,因变量为单条话题平均在榜时长:
\begin{equation}
\text{Density}_{it} = \beta_0 + \beta_1 \text{Post}_t + \beta_2 \text{Trend}_t + \beta_3 (\text{Post}_t \times \text{Trend}_t) + \varepsilon_t
\end{equation}
根据H4a和H4b,预期社会类的$\beta_1 \geq 0$(密度保持稳定或上升,``少而精''),娱乐类的$\beta_1 > 0$(密度上升,``多且长'')。

以上模型设定充分利用了政策冲击的准自然实验特征,通过OLS时间序列分析和DID的结合使用,从整体层面和类别内部两个维度识别算法治理的因果效应,为后续实证分析提供了严谨的计量框架。
