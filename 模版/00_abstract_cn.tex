% 中文摘要
\begin{abstract}
算法推荐系统作为数字平台的核心信息配置机制,在优化用户体验的同时也引发了信息茧房、注意力垄断等公共治理难题。如何在释放算法价值的同时有效规制其负外部性,成为全球数字治理的核心议题。本文利用2024年11月中国清朗行动算法治理政策这一准自然实验,采用断点回归时间序列与双重差分相结合的识别策略,基于微博热搜榜高频数据系统评估算法治理对平台内容生态的重构效应。研究发现,算法治理有效推动了注意力分布的去中心化,瓦解了头部话题的垄断格局;分级分类治理导致社会类内容质量显著提升而娱乐类保持稳定,形成差异化的质量改善路径。然而,在注意力总量守恒的零和约束下,监管对高风险领域的强化治理引发了注意力向低风险领域的系统性转移,呈现典型的``水床效应''。在密度层面,社会类呈``少而精''调整,娱乐类呈``多且长''扩张。本研究揭示了算法治理的复杂效应:多样性提升与质量改善的积极成效,与注意力再分配的非预期后果并存,形成``监管初衷''与``生态演化''之间的张力。研究为理解算法治理的系统性后果提供了新的实证证据,对完善分级分类治理体系、探索``激励优质内容+动态监测调整+跨平台协同治理''的系统性框架具有重要政策启示。
\end{abstract}

\keywords{算法治理;平台生态系统;注意力经济;断点回归时间序列;水床效应}

\noindent JEL分类号:D83;L51;L86
