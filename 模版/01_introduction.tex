\section{引言}

数字平台已成为信息传播、社会互动和经济活动的核心基础设施。作为平台生态系统的``神经中枢''\citep{Jacobides2018},算法推荐系统在优化用户体验、提升商业价值的同时,也在深刻塑造着公共信息环境。当算法推荐逐渐成为亿万用户获取信息的``把关人''\citep{Gorwa2020},其带来的信息茧房、注意力垄断、内容低质化等问题日益凸显。平台算法通过个性化推荐强化用户既有偏好,可能导致``过滤气泡''效应\citep{Pariser2011};头部内容凭借算法加权获得不成比例的曝光,形成``赢者通吃''的马太效应\citep{Hindman2018};为追求点击率最大化,诱导性标题(clickbait)泛滥,信息质量持续下滑\citep{Chen2015}。如何有效治理算法推荐系统,已成为全球数字治理的核心议题。

面对算法推荐带来的系统性挑战,各国监管机构纷纷探索治理路径。欧盟《数字服务法》(DSA)要求大型平台披露算法推荐逻辑并提供``无算法''选项;美国学界和政策界围绕《算法问责法案》展开激烈讨论;中国在算法治理领域走在世界前列,《互联网信息服务算法推荐管理规定》(2022年)和《生成式人工智能服务管理暂行办法》(2023年)构建了全球最为系统的算法监管框架\citep{Creemers2022}。2024年11月12日,国家网信办启动新一轮清朗行动,聚焦算法推荐治理,发布《算法推荐服务专项治理清单指引》,明确27项核验标准,要求各大平台``不得利用算法操纵热点话题''、``保障信息内容多样性''、``降低低俗内容权重'',并强化分级分类管理,对社会类等高风险内容实施更严格的审核机制。这一政策冲击为学术界提供了难得的准自然实验场景,引发了本文关注的核心现实问题:当外部监管力量介入平台算法系统,内容生态将如何响应?监管目标能否实现?是否会产生非预期后果?

现有文献在理论层面存在两个关键缺口,制约了对上述问题的系统性回答。第一,缺乏统一的理论框架解释监管引发的多维效应。尽管平台治理研究已积累丰富成果——涵盖平台生态系统理论\citep{Tiwana2010}、平台边界资源管理\citep{Ghazawneh2013}、多边市场治理\citep{Huber2013, Parker2018}等多个分支——但既有研究多聚焦于单一治理目标(如内容质量或多样性),缺乏将多元治理效应纳入统一分析框架的理论尝试。特别是,算法治理如何同时影响内容质量、注意力分布和生态结构,这些效应之间存在何种内在关联,现有研究尚未提供系统性解释。第二,忽视了监管的非预期后果与目标冲突。注意力是数字平台最核心的稀缺资源\citep{Davenport2001, Webster2014},用户的总注意力预算在短期内近似固定,形成典型的零和博弈格局。在此约束下,对某一内容类型的监管强化可能引发注意力在不同类型间的系统性再分配。然而,现有算法治理文献\citep{Gritsenko2022, Ulbricht2022}多停留在规范性讨论,缺乏对监管溢出效应的理论刻画。这引出了本文关注的核心理论问题:在资源约束条件下,对高风险内容的监管强化是否会触发平台的资源再配置行为,进而产生``按下葫芦浮起瓢''的水床效应?实证层面同样面临挑战。由于算法系统的``黑箱''特性和高频政策调整,学界难以获得清晰的政策断点和高质量数据,导致算法治理效果的因果识别研究极为稀缺\citep{Jhaver2019}。

本文聚焦2024年11月清朗行动算法推荐治理这一准自然实验,以微博热搜榜为研究对象,采用断点回归时间序列(ITS)识别主效应,辅以双重差分(DID)进行稳健性检验,基于覆盖政策前后共计608天、超过25万条观测值的高频数据,系统评估算法治理对平台内容生态的多维重构效应。本文构建了基于资源依赖理论的统一分析框架,将用户注意力视为平台依赖的核心稀缺资源,系统分析算法治理如何通过改变不同内容类型的资源获取成本,触发平台的资源再配置行为,进而产生多维系统效应。研究发现:(1)内容质量分化——分级分类治理导致社会类内容质量显著提升(信息差指数下降20.0\%、机构媒体占比上升25.2\%),而娱乐类质量保持稳定,DID估计显示机构媒体相对自媒体的信息差净下降2.05个百分点;(2)注意力去中心化——算法治理打破了头部垄断格局(Shannon熵上升18.7\%,HHI下降11.8\%,单条话题重复上榜次数下降23.4\%);(3)水床效应——在注意力总量守恒的零和约束下($p=0.63$),社会类份额即时下降3.16个百分点,娱乐类份额上升3.84个百分点,呈现``社会类紧缩—娱乐类扩张''的跷跷板格局;(4)密度分化——社会类呈``少而精''调整(话题数减少、单条时长增加),娱乐类呈``多且长''调整。本文的理论贡献在于构建了基于资源依赖理论的算法治理分析框架,提出``水床效应''机制揭示监管的间接传导路径;实证贡献在于利用准自然实验提供了算法治理因果效应的可信估计,为政策评估文献增添了中国证据;政策贡献在于揭示了算法治理中多元目标的潜在冲突——质量提升与多样性改善的同时,可能伴随公共议题曝光度的下降——提示监管部门需要建立更加系统化的治理策略。本文的结构安排如下:第二部分梳理相关文献并构建理论框架,推导研究假说;第三部分介绍研究设计和数据来源;第四部分报告实证结果及稳健性检验;第五部分讨论研究发现的理论含义和政策启示。
