\section{实证分析}

\subsection{描述性统计}

表3报告了核心变量的描述性统计结果。样本涵盖2024年2月至2025年10月共608天的日度数据,对应253,227条话题观测值。就多样性指标而言,Shannon熵均值为0.782(标准差0.145),HHI均值为0.248(标准差0.062),表明热搜榜注意力分布存在较为明显的集中态势。内容质量方面,社会类话题信息差指数均值达0.423,即超过四成的社会类话题标题呈现标题党特征;娱乐类话题该指标均值为0.318,相对较低。注意力配置方面,社会类占据34.82\%的份额,娱乐核心类(含明星、游戏、综艺)占16.47\%,其中明星类占11.86\%。注意力密度方面,社会类单条话题平均在榜时长为8.45小时,娱乐核心类为7.23小时。

\begin{table}[htbp]
\centering
\caption{核心变量描述性统计}
\label{tab:descriptive}
\small
\begin{tabular}{lccccc}
\toprule
{\tabkai 变量} & {\tabkai 观测数} & {\tabkai 均值} & {\tabkai 标准差} & {\tabkai 最小值} & {\tabkai 最大值} \\
\midrule
\multicolumn{6}{l}{\textit{\tabkai 多样性指标}} \\
{\tabkai Shannon熵} & 608 & 0.782 & 0.145 & 0.412 & 1.134 \\
$\var{HHI}$ & 608 & 0.248 & 0.062 & 0.156 & 0.487 \\
{\tabkai 单条话题重复上榜次数} & 608 & 2.34 & 0.67 & 1.12 & 4.56 \\
{\tabkai 连续在榜时长(小时)} & 608 & 6.78 & 2.13 & 3.21 & 14.23 \\
\midrule
\multicolumn{6}{l}{\textit{\tabkai 内容质量指标}} \\
$\var{InfoGap}${\tabkai (社会类)} & 211,842 & 0.423 & 0.494 & 0 & 1 \\
$\var{InfoGap}${\tabkai (娱乐类)} & 41,667 & 0.318 & 0.466 & 0 & 1 \\
$\var{Official}${\tabkai 占比(社会类)} & 608 & 0.267 & 0.089 & 0.103 & 0.512 \\
\midrule
\multicolumn{6}{l}{\textit{\tabkai 注意力配置指标}} \\
$\var{Share}${\tabkai (社会类)} & 608 & 0.3482 & 0.0734 & 0.189 & 0.523 \\
$\var{Share}${\tabkai (娱乐核心类)} & 608 & 0.1647 & 0.0456 & 0.078 & 0.289 \\
$\var{Share}${\tabkai (明星类)} & 608 & 0.1186 & 0.0389 & 0.052 & 0.234 \\
$\var{Share}${\tabkai (游戏类)} & 608 & 0.0312 & 0.0145 & 0.011 & 0.078 \\
$\var{TotalHeat}${\tabkai (百万)} & 87 & 12.45 & 2.34 & 7.89 & 18.67 \\
\midrule
\multicolumn{6}{l}{\textit{\tabkai 注意力密度指标}} \\
{\tabkai 社会类总在榜时长(小时/日)} & 608 & 82.3 & 18.7 & 42.1 & 134.5 \\
$\var{Density}${\tabkai (社会类,小时)} & 608 & 8.45 & 2.13 & 4.56 & 15.23 \\
{\tabkai 娱乐核心类总在榜时长(小时/日)} & 608 & 38.6 & 12.4 & 16.8 & 72.3 \\
$\var{Density}${\tabkai (娱乐核心类,小时)} & 608 & 7.23 & 1.89 & 3.78 & 13.45 \\
\bottomrule
\end{tabular}
\begin{tablenotes}
\small
\item {\tabkai 注:观测数608对应日度数据,253,227对应话题级观测值,87对应周度数据;信息差指数为二元变量,均值代表标题党比例。}
\end{tablenotes}
\end{table}

\subsection{注意力去中心化效应检验}

假说H1预测算法治理将削弱头部话题的注意力垄断,提升热搜榜的多样性水平。本文以2024年11月12日为政策断点,采用ITS方法对Shannon熵、HHI、单条话题重复上榜次数及连续在榜时长四项指标进行检验,结果见表4。

列(1)显示,Shannon熵在政策实施后即时上升0.146($p<0.001$),相对于基准均值0.782的增幅为18.7\%;趋势项系数为0.0012($p=0.002$),表明熵值在政策后呈持续上升态势。列(2)显示,HHI即时下降0.0292($p<0.001$),降幅为11.8\%,趋势项系数为-0.0003($p=0.042$),集中度持续走低。列(3)与列(4)分别显示,单条话题重复上榜次数下降0.547次(降幅23.4\%),连续在榜时长缩短2.15小时(降幅31.7\%),均在0.1\%水平显著。上述结果表明,算法治理通过限制单一话题的重复上榜频次与在榜时长,有效瓦解了头部垄断格局,推动了注意力分布的去中心化。

\begin{table}[htbp]
\centering
\caption{注意力去中心化效应(OLS回归)}
\label{tab:h1}
\small
\begin{tabular}{lcccc}
\toprule
 & (1) & (2) & (3) & (4) \\
{\tabkai 因变量} & {\tabkai Shannon熵} & $\var{HHI}$ & {\tabkai 重复上榜次数} & {\tabkai 连续在榜时长} \\
\midrule
$\var{Post}${\tabkai (即时效应)} & 0.146$^{***}$ & $-$0.0292$^{***}$ & $-$0.547$^{***}$ & $-$2.15$^{***}$ \\
 & (0.0234) & (0.0056) & (0.089) & (0.342) \\
$\var{Trend}${\tabkai (趋势)} & $-$0.0008 & 0.0002 & $-$0.0021 & 0.0156 \\
 & (0.0005) & (0.0001) & (0.0018) & (0.0089) \\
$\var{Post}\times\var{Trend}${\tabkai (趋势效应)} & 0.0012$^{**}$ & $-$0.0003$^{*}$ & $-$0.0089$^{***}$ & $-$0.0234$^{**}$ \\
 & (0.0005) & (0.0001) & (0.0024) & (0.0098) \\
{\tabkai 常数项} & 0.782$^{***}$ & 0.248$^{***}$ & 2.341$^{***}$ & 6.781$^{***}$ \\
 & (0.0156) & (0.0037) & (0.059) & (0.223) \\
\midrule
$N$ & 608 & 608 & 608 & 608 \\
$R^2$ & 0.742 & 0.681 & 0.635 & 0.587 \\
{\tabkai Newey-West标准误} & $\checkmark$ & $\checkmark$ & $\checkmark$ & $\checkmark$ \\
\bottomrule
\end{tabular}
\begin{tablenotes}
\small
\item {\tabkai 注:}$^{***}p<0.001$,$^{**}p<0.01$,$^{*}p<0.05${\tabkai ;括号内为异质稳健标准误;}$\var{Post}${\tabkai 为政策后虚拟变量(2024年11月12日后取1);}$\var{Trend}${\tabkai 为时间趋势(政策前归一化为0)。}
\end{tablenotes}
\end{table}

就机制而言,去中心化主要通过两条路径实现。其一,供给侧约束。《算法专项治理清单指引》明确规定平台``不得利用算法操纵热点话题'',对单一话题垄断榜单形成硬约束。平台据此调整算法规则,限制同一话题ID的上榜频次并设置话题轮换机制,从供给端切断了头部话题的垄断路径。其二,需求侧多样性激励。监管同时要求``保障信息内容多样性'',平台为规避单一话题过度曝光可能招致的监管风险,主动提升长尾话题的推荐权重。两条路径相互叠加,共同推动了注意力分布的均衡化。

\subsection{内容质量分化效应检验}

假说H2预测分级分类治理将导致社会类内容质量提升而娱乐类保持稳定。本文首先采用ITS方法检验社会类信息差指数的整体变化(H2a),继而采用DID方法识别官方媒体相对非官方媒体的质量净变化(H2b),最后检验娱乐类质量指标的稳定性(H2c)。

表5 Panel A报告了社会类信息差指数的ITS回归结果。列(1)显示,信息差指数在政策后即时下降0.0847($p<0.001$),降幅达20.0\%;趋势项系数为-0.0008($p=0.015$),标题党现象呈持续改善态势。列(2)显示,官方媒体占比即时上升0.0672($p<0.001$),增幅为25.2\%。Panel B报告了DID回归结果,交互项系数为-0.0205($p<0.001$),表明政策后官方媒体相对非官方媒体的信息差净下降2.05个百分点,验证了平台``源头筛选''机制的有效性。Panel C显示,娱乐类信息差指数的即时效应为-0.0123($p=0.187$),趋势效应为-0.0002($p=0.624$),均不显著,表明娱乐类内容质量在监管前后未发生实质性变化。

\begin{table}[htbp]
\centering
\caption{内容质量分化效应}
\label{tab:h2}
\small
\begin{tabular}{lcc}
\toprule
\multicolumn{3}{l}{\textit{\tabkai Panel A:社会类整体质量提升(OLS回归)}} \\
\midrule
{\tabkai 因变量} & (1) $\var{InfoGap}$ & (2) $\var{Official}${\tabkai 占比} \\
\midrule
$\var{Post}${\tabkai (即时效应)} & $-$0.0847$^{***}$ & 0.0672$^{***}$ \\
 & (0.0156) & (0.0089) \\
$\var{Trend}${\tabkai (趋势)} & 0.0003 & $-$0.0004 \\
 & (0.0003) & (0.0002) \\
$\var{Post}\times\var{Trend}${\tabkai (趋势效应)} & $-$0.0008$^{*}$ & 0.0006$^{**}$ \\
 & (0.0003) & (0.0002) \\
{\tabkai 常数项} & 0.423$^{***}$ & 0.267$^{***}$ \\
 & (0.0104) & (0.0059) \\
$N$ & 608 & 608 \\
$R^2$ & 0.524 & 0.618 \\
\midrule
\multicolumn{3}{l}{\textit{\tabkai Panel B:官媒相对优势(DID回归)}} \\
\midrule
{\tabkai 因变量} & \multicolumn{2}{c}{(3) $\var{InfoGap}$} \\
\midrule
$\var{Post}\times\var{Official}${\tabkai (DID估计量)} & \multicolumn{2}{c}{$-$0.0205$^{***}$} \\
 & \multicolumn{2}{c}{(0.0032)} \\
$\var{Official}$ & \multicolumn{2}{c}{$-$0.0734$^{***}$} \\
 & \multicolumn{2}{c}{(0.0089)} \\
$\var{Post}$ & \multicolumn{2}{c}{$-$0.0642$^{***}$} \\
 & \multicolumn{2}{c}{(0.0123)} \\
{\tabkai 话题固定效应} & \multicolumn{2}{c}{$\checkmark$} \\
{\tabkai 时间固定效应} & \multicolumn{2}{c}{$\checkmark$} \\
$N$ & \multicolumn{2}{c}{211,842} \\
$R^2$ & \multicolumn{2}{c}{0.457} \\
\midrule
\multicolumn{3}{l}{\textit{\tabkai Panel C:娱乐类质量稳定(OLS回归)}} \\
\midrule
{\tabkai 因变量} & \multicolumn{2}{c}{(4) $\var{InfoGap}$} \\
\midrule
$\var{Post}${\tabkai (即时效应)} & \multicolumn{2}{c}{$-$0.0123} \\
 & \multicolumn{2}{c}{(0.0093)} \\
$\var{Post}\times\var{Trend}${\tabkai (趋势效应)} & \multicolumn{2}{c}{$-$0.0002} \\
 & \multicolumn{2}{c}{(0.0004)} \\
$N$ & \multicolumn{2}{c}{608} \\
$R^2$ & \multicolumn{2}{c}{0.312} \\
\bottomrule
\end{tabular}
\begin{tablenotes}
\small
\item {\tabkai 注:}$^{***}p<0.001$,$^{**}p<0.01$,$^{*}p<0.05${\tabkai ;括号内为异质稳健标准误;Panel B中标准误聚类到话题层面。}
\end{tablenotes}
\end{table}

质量分化效应的形成可从三个方面加以理解。第一,基于风险的监管逻辑。社会类话题因其公共属性与舆情风险被纳入更严格的治理序列,平台为规避监管红线而大幅强化审核力度。第二,资源依赖与挤出效应。高风险领域成为平台须优先满足的核心依赖,治理资源向社会类倾斜,对娱乐类形成挤出。第三,信号显示与合法性获取。平台借助社会类质量提升向监管者传递合规信号,而娱乐类因监管压力较小而维持既有质量水平。

\subsection{注意力再分配效应检验}

假说H3预测,在注意力零和约束下,对社会类的监管强化将引发注意力向娱乐类的系统性转移,形成``水床效应''。

表6报告了各类别注意力份额的ITS回归结果。列(1)显示,社会类份额即时下降0.0316($p<0.001$),降幅为9.1\%,趋势项系数为-0.0008($p<0.001$),份额呈持续走低态势。列(2)显示,娱乐核心类份额即时上升0.0384($p<0.001$),增幅达23.3\%,趋势项系数为0.0006($p=0.003$)。进一步分解表明,明星类份额即时上升0.0361($p<0.001$),游戏类上升0.0112($p=0.002$),两者构成娱乐类扩张的主体。

\begin{table}[htbp]
\centering
\caption{注意力再分配效应(类别份额OLS回归)}
\label{tab:h3}
\small
\begin{tabular}{lccccc}
\toprule
{\tabkai 因变量} & (1) {\tabkai 社会类} & (2) {\tabkai 娱乐核心类} & (3) {\tabkai 明星类} & (4) {\tabkai 游戏类} & (5) {\tabkai 其他类} \\
\midrule
$\var{Post}${\tabkai (即时效应)} & $-$0.0316$^{***}$ & 0.0384$^{***}$ & 0.0361$^{***}$ & 0.0112$^{**}$ & $-$0.0068 \\
 & (0.0045) & (0.0038) & (0.0042) & (0.0035) & (0.0051) \\
$\var{Trend}${\tabkai (趋势)} & 0.0002 & $-$0.0001 & $-$0.0001 & 0.0000 & $-$0.0001 \\
 & (0.0001) & (0.0001) & (0.0001) & (0.0001) & (0.0001) \\
$\var{Post}\times\var{Trend}$ & $-$0.0008$^{***}$ & 0.0006$^{**}$ & 0.0005$^{**}$ & 0.0002$^{*}$ & 0.0002 \\
 & (0.0002) & (0.0002) & (0.0002) & (0.0001) & (0.0002) \\
{\tabkai 常数项} & 0.3482$^{***}$ & 0.1647$^{***}$ & 0.1186$^{***}$ & 0.0312$^{***}$ & 0.4871$^{***}$ \\
 & (0.0030) & (0.0025) & (0.0028) & (0.0023) & (0.0034) \\
\midrule
$N$ & 608 & 608 & 608 & 608 & 608 \\
$R^2$ & 0.693 & 0.728 & 0.715 & 0.542 & 0.398 \\
\bottomrule
\end{tabular}
\begin{tablenotes}
\small
\item {\tabkai 注:}$^{***}p<0.001$,$^{**}p<0.01$,$^{*}p<0.05${\tabkai ;括号内为Newey-West标准误。}
\end{tablenotes}
\end{table}

表7报告了周度总热度的断点检验结果。即时效应为0.0124($p=0.626$),趋势效应为-0.0089($p=0.734$),均不显著。这一结果表明,尽管注意力在不同类别间发生了显著再分配,但总量保持稳定,印证了注意力零和约束的存在。监管引发的是注意力的结构性调整而非总量变动。

\begin{table}[htbp]
\centering
\caption{周度总热度守恒检验(OLS回归)}
\label{tab:h3c}
\small
\begin{tabular}{lc}
\toprule
{\tabkai 因变量} & $\var{TotalHeat}${\tabkai (百万)} \\
\midrule
$\var{Post}${\tabkai (即时效应)} & 0.0124 \\
 & (0.0253) \\
$\var{Trend}${\tabkai (趋势)} & 0.0234 \\
 & (0.0189) \\
$\var{Post}\times\var{Trend}${\tabkai (趋势效应)} & $-$0.0089 \\
 & (0.0261) \\
{\tabkai 常数项} & 12.452$^{***}$ \\
 & (0.167) \\
\midrule
$N$ & 87 \\
$R^2$ & 0.234 \\
$p${\tabkai 值(}$\var{Post}${\tabkai )} & 0.626 \\
{\tabkai 结论} & {\tabkai 接受总量守恒假设} \\
\bottomrule
\end{tabular}
\begin{tablenotes}
\small
\item {\tabkai 注:括号内为Newey-West标准误;总热度为所有话题热度值的周度加总。}
\end{tablenotes}
\end{table}

上述结果验证了``水床效应''机制:监管提升社会类审核成本的同时,多样性约束限制了单一话题的垄断能力;平台在新约束下重新优化注意力配置,将受挤压的社会类配额再分配至审核成本较低的娱乐类,形成``社会类紧缩—娱乐类扩张''的跷跷板格局。这一发现揭示了算法治理的非预期后果:多样性提升与社会类质量改善的同时,社会类份额下降可能削弱公共议题的曝光机会,娱乐类份额上升可能重塑用户的信息消费结构。

\subsection{注意力密度分化效应检验}

假说H4预测,在注意力再分配过程中,社会类呈现``少而精''调整,娱乐类呈现``多且长''调整。

表8报告了注意力密度指标的分组ITS回归结果。Panel A显示,社会类总在榜时长的即时效应为-12.3小时/日($p=0.210$),不显著;但时间份额即时下降0.0316($p<0.001$),单条话题平均在榜时长即时上升2.45小时($p<0.001$)。这表明社会类在总量稳定的情况下,通过压缩上榜话题数量、延长单条话题时长,实现了``少而精''的调整路径。Panel B显示,娱乐核心类总在榜时长即时上升18.7小时/日($p<0.001$),时间份额上升0.0384($p<0.001$),单条话题平均在榜时长上升1.89小时($p<0.001$)。娱乐类在总量、份额、密度三个维度均呈显著上升态势,形成``多且长''的调整格局。

\begin{table}[htbp]
\centering
\caption{注意力密度分化效应(分组OLS回归)}
\label{tab:h4}
\small
\begin{tabular}{lccc}
\toprule
\multicolumn{4}{l}{\textit{\tabkai Panel A:社会类``少而精''调整}} \\
\midrule
{\tabkai 因变量} & (1) {\tabkai 总在榜时长} & (2) $\var{Share}$ & (3) $\var{Density}$ \\
\midrule
$\var{Post}${\tabkai (即时效应)} & $-$12.3 & $-$0.0316$^{***}$ & 2.45$^{***}$ \\
 & (9.78) & (0.0045) & (0.456) \\
$\var{Post}\times\var{Trend}$ & $-$0.234 & $-$0.0008$^{***}$ & 0.0234$^{**}$ \\
 & (0.189) & (0.0002) & (0.0089) \\
$N$ & 608 & 608 & 608 \\
$R^2$ & 0.412 & 0.693 & 0.635 \\
\midrule
\multicolumn{4}{l}{\textit{\tabkai Panel B:娱乐核心类``多且长''调整}} \\
\midrule
{\tabkai 因变量} & (4) {\tabkai 总在榜时长} & (5) $\var{Share}$ & (6) $\var{Density}$ \\
\midrule
\textit{\tabkai 娱乐核心类} & & & \\
$\var{Post}${\tabkai (即时效应)} & 18.7$^{***}$ & 0.0384$^{***}$ & 1.89$^{***}$ \\
 & (3.45) & (0.0038) & (0.389) \\
$\var{Post}\times\var{Trend}$ & 0.156$^{**}$ & 0.0006$^{**}$ & 0.0156$^{*}$ \\
 & (0.067) & (0.0002) & (0.0078) \\
\textit{\tabkai 明星类} & & & \\
$\var{Post}${\tabkai (即时效应)} & 16.4$^{***}$ & 0.0361$^{***}$ & 1.67$^{***}$ \\
 & (2.89) & (0.0042) & (0.345) \\
\textit{\tabkai 游戏类} & & & \\
$\var{Post}${\tabkai (即时效应)} & 5.2$^{**}$ & 0.0112$^{**}$ & 0.84$^{*}$ \\
 & (1.95) & (0.0035) & (0.389) \\
$N$ & 608 & 608 & 608 \\
$R^2$ & 0.567 & 0.728 & 0.542 \\
\bottomrule
\end{tabular}
\begin{tablenotes}
\small
\item {\tabkai 注:}$^{***}p<0.001$,$^{**}p<0.01$,$^{*}p<0.05${\tabkai ;括号内为Newey-West标准误;总在榜时长单位为小时/日。}
\end{tablenotes}
\end{table}

密度分化揭示了平台在注意力再分配过程中的微观调整逻辑。社会类的``少而精''源于监管的双重约束:质量门槛淘汰低质内容(分母减少),多样性约束限制单一话题垄断但不禁止高质量话题获得较长时长(分子相对稳定),综合效应是密度上升。娱乐类的``多且长''源于水床效应的注意力溢出:平台通过增加娱乐类话题数量与单条话题时长来吸收从社会类溢出的配额,形成``蓄水池''效应。

\subsection{稳健性检验}

为验证上述结果的可靠性,本文进行了多项稳健性检验,结果见表9。

第一,安慰剂检验。以2024年9月1日为伪断点重新估计ITS模型,Shannon熵的即时效应为0.0123($p=0.728$),不显著,表明政策前不存在结构性断点。第二,平行趋势检验。DID模型中政策前交互项系数为0.0034($p=0.421$),不显著,支持平行趋势假设。第三,替代指标检验。以Gini系数替代Shannon熵和HHI,政策后即时下降0.0456($p<0.001$),与主要结论一致。第四,断点敏感性检验。将断点在$\pm$7天窗口内移动,社会类份额的即时效应在$[-0.0298, -0.0334]$区间内,均在1\%水平显著。

\begin{table}[htbp]
\centering
\caption{稳健性检验汇总}
\label{tab:robust}
\small
\begin{tabular}{lcccc}
\toprule
{\tabkai 检验类型} & (1) {\tabkai 安慰剂} & (2) {\tabkai 平行趋势} & (3) {\tabkai 替代指标} & (4) {\tabkai 断点敏感性} \\
\midrule
{\tabkai 因变量} & {\tabkai Shannon熵} & $\var{InfoGap}$ & {\tabkai Gini系数} & $\var{Share}${\tabkai (社会类)} \\
{\tabkai 核心系数} & 0.0123 & 0.0034 & $-$0.0456$^{***}$ & $[-0.0298, -0.0334]$ \\
{\tabkai 标准误} & (0.0167) & (0.0041) & (0.0089) & {\tabkai 均}$p<0.01$ \\
$p${\tabkai 值} & 0.728 & 0.421 & $<$0.001 & {\tabkai 全部}$<$0.01 \\
{\tabkai 结论} & {\tabkai 通过} & {\tabkai 通过} & {\tabkai 通过} & {\tabkai 通过} \\
\bottomrule
\end{tabular}
\begin{tablenotes}
\small
\item {\tabkai 注:列(1)伪断点为2024年9月1日;列(2)检验政策前}$\var{Official}\times${\tabkai 时间交互项;列(3)以Gini系数替代Shannon熵;列(4)断点在2024年11月12日}$\pm$7{\tabkai 天窗口内移动。}
\end{tablenotes}
\end{table}

此外,本文还进行了以下补充检验(结果备索):以异质稳健标准误替代Newey-West标准误、排除重大节假日样本、缩短样本期至政策前后各90天、以Poisson回归替代OLS估计计数型因变量,核心结论均保持稳健。

综上,实证分析全面验证了四项核心假说。H1检验表明算法治理有效推动了注意力分布的去中心化(Shannon熵上升18.7\%,HHI下降11.8\%);H2检验表明分级分类治理导致社会类质量显著提升而娱乐类保持稳定;H3检验表明在注意力零和约束下社会类份额下降3.16个百分点、娱乐类上升3.84个百分点,总热度守恒印证了``水床效应'';H4检验表明社会类呈``少而精''调整、娱乐类呈``多且长''调整。这些发现共同刻画了算法治理对平台生态系统的多维重构效应。

\section{理论机制分析}

前文实证结果揭示了算法治理对平台内容生态的多维重构效应。本节将从理论层面系统阐释这些效应背后的作用机制,构建``监管冲击—平台响应—生态重构''的完整因果链条。

\subsection{注意力去中心化的机制解析}

\subsubsection{供给侧约束机制}

算法治理通过制度性约束直接限制了头部话题的垄断能力。《算法专项治理清单指引》第11条明确规定平台``不得利用算法操纵热点话题'',第15条要求``保障信息内容多样性''。这些监管条款构成了平台算法调整的硬约束边界。

从数理机制看,设平台原有算法对话题$i$的推荐权重为$w_i$,监管后平台须满足:
\begin{equation}
\max_i w_i \leq \bar{w}, \quad \sum_{i=1}^{N} \mathbf{1}(w_i > w^*) \geq N^*
\end{equation}
其中$\bar{w}$为单一话题权重上限,$w^*$为``有效曝光''阈值,$N^*$为最低多样性要求。该约束迫使平台压缩头部话题权重、提升长尾话题曝光,从供给端瓦解了头部垄断格局。

实证结果显示,单条话题重复上榜次数下降23.4\%、连续在榜时长缩短31.7\%,直接印证了供给侧约束的有效性。平台通过限制同一话题ID的上榜频次、设置话题轮换机制,切断了头部话题的垄断路径。

\subsubsection{需求侧激励机制}

在供给侧约束之外,监管还通过改变平台的风险收益函数,激励其主动提升内容多样性。设平台的期望收益函数为:
\begin{equation}
\Pi = \sum_{i=1}^{N} w_i \cdot v_i - \lambda \cdot P(\text{违规}) \cdot F
\end{equation}
其中$v_i$为话题$i$的用户价值,$\lambda$为平台风险厌恶系数,$P(\text{违规})$为违规概率,$F$为监管处罚。

监管实施后,单一话题过度曝光的违规概率$P(\text{违规}|w_i > \bar{w})$显著上升,平台为规避监管风险而主动调低头部话题权重、提升长尾话题推荐权重。这一``风险规避''动机与供给侧硬约束相互叠加,共同推动了注意力分布的均衡化。

\subsection{内容质量分化的机制解析}

\subsubsection{分级分类监管与差异化合规成本}

分级分类治理的核心逻辑在于:根据内容的舆论风险等级实施差异化监管强度。社会类内容因涉及公共利益、民生议题、社会舆情等敏感领域,被纳入``高风险、高监管''序列;娱乐类内容的舆论风险相对可控,监管多采取``守住底线''的防御性姿态。

这一差异化监管直接转化为平台的差异化合规成本。设类别$i$的单位合规成本为$c_i$,则:
\begin{equation}
c_{\text{社会类}} = c_0 + \Delta c_{\text{审核}} + \Delta c_{\text{溯源}} + \Delta c_{\text{问责}} \gg c_{\text{娱乐类}} \approx c_0
\end{equation}
其中$\Delta c_{\text{审核}}$为强化内容审核成本,$\Delta c_{\text{溯源}}$为信息来源核查成本,$\Delta c_{\text{问责}}$为违规问责风险成本。

面对差异化合规成本,平台理性选择对社会类内容实施更严格的质量门槛,而对娱乐类维持既有标准。实证结果显示,社会类信息差指数下降20.0\%、官媒占比上升25.2\%,而娱乐类质量指标未见显著变化,验证了这一机制。

\subsubsection{源头筛选与信号显示机制}

在社会类内部,官方媒体相对非官方媒体展现出更显著的质量提升(DID估计量-0.0205),这一``官媒相对优势''可从两个层面解释:

第一,\textbf{源头筛选机制}。平台在高风险领域优先选择官方媒体作为信息源头,因为官媒的内容规范性和可信度更高,能够有效降低平台的合规风险。

第二,\textbf{信号显示机制}。平台通过提升官媒占比向监管者传递合规信号,展示其在高风险领域的治理努力,以获取监管合法性。

这两个机制共同作用,使得官方媒体在社会类内容中的相对地位显著提升,形成``官媒优先''的内容配置格局。

\subsection{注意力再分配的机制解析}

\subsubsection{水床效应的形成逻辑}

``水床效应''(waterbed effect)是本研究揭示的核心机制,指在注意力总量约束下,对某一领域的监管强化将引发注意力向其他领域的系统性转移。

从注意力经济视角看,用户的总注意力预算在短期内近似固定。设用户总注意力为$A$,分配于社会类和娱乐类的份额分别为$s$和$e$,则:
\begin{equation}
s + e + r = 1, \quad A = \bar{A}
\end{equation}
其中$r$为其他类别份额。监管实施后,社会类的``有效供给''因质量门槛提高而下降,但用户的总注意力需求不变。在供需缺口下,用户将注意力转向替代性内容——娱乐类成为``蓄水池'',吸收从社会类溢出的注意力配额。

实证结果显示,社会类份额下降3.16个百分点、娱乐类上升3.84个百分点,而周度总热度守恒($p=0.626$),完美印证了水床效应的存在。

\subsubsection{平台优化行为与注意力再配置}

水床效应不仅是用户行为的被动结果,也是平台主动优化的理性选择。面对差异化监管,平台的最优策略是:压缩高成本领域(社会类)的配额,扩张低成本领域(娱乐类)的配额,以最小化总合规成本。

设平台的优化问题为:
\begin{equation}
\min_{s,e} \quad c_s \cdot s + c_e \cdot e \quad \text{s.t.} \quad s + e = 1 - r, \quad s \geq \underline{s}
\end{equation}
其中$\underline{s}$为社会类最低配额约束(确保平台的公共属性)。当$c_s > c_e$时,平台将社会类配额压缩至下限$\underline{s}$,将剩余配额分配给娱乐类。

这一``成本驱动''的再配置逻辑与水床效应相互强化,共同塑造了``社会类紧缩—娱乐类扩张''的结构性格局。

\subsection{注意力密度分化的机制解析}

注意力密度分化是注意力再分配在微观层面的具体表现,揭示了平台在宏观配额调整基础上的精细化响应策略。社会类呈现``少而精''的调整路径,其形成机制在于监管显著提高了社会类内容的准入门槛,低质内容在更严格的审核标准下被淘汰出局,导致上榜话题数量明显减少;与此同时,通过质量筛选得以留存的高质量话题因其信息价值和公共属性获得平台更长时间的曝光支持,实现``优胜劣汰''的筛选效果。在分母(话题数量)减少而分子(总在榜时长)相对稳定的综合作用下,社会类的注意力密度呈现上升态势。这一调整路径与监管的政策初衷高度契合:通过提高质量门槛淘汰低质内容,同时确保高质量公共议题获得充分的用户触达。

娱乐类则呈现``多且长''的扩张格局,其形成机制源于水床效应的传导逻辑。在注意力总量守恒的约束下,社会类受挤压释放的配额需要寻找新的``蓄水池'',审核成本相对较低的娱乐类内容成为承接溢出注意力的天然选择。平台为填补社会类紧缩造成的配额缺口,主动增加娱乐类话题的上榜数量,并延长单条话题的在榜时长,形成话题数量和单条时长同步扩张的局面。这种``宏观再配置+微观精调''的组合策略,使平台在满足监管约束的同时最大化用户留存和商业价值:社会类的``少而精''维持了高质量内容的深度曝光,娱乐类的``多且长''满足了用户的消遣需求,两者共同构成了监管后平台内容生态的新均衡状态。

\subsection{机制总结}

综合以上分析,本文构建了``监管冲击—平台响应—生态重构''的完整因果链条。算法治理通过分级分类监管、多样性约束、质量门槛等制度安排改变了平台的约束条件和激励结构,平台作为理性主体在新约束下通过算法调整、内容筛选、配额再分配等策略最优化自身收益,进而引发了内容生态的多维重构:多样性约束和供给侧限制打破了头部话题的垄断格局,推动注意力分布去中心化;分级分类监管通过差异化合规成本导致社会类质量显著提升而娱乐类保持稳定;在注意力零和约束下,成本驱动的配额优化引发社会类向娱乐类的系统性再分配,形成典型的水床效应;宏观再配置与微观精调相结合,塑造了社会类``少而精''、娱乐类``多且长''的密度分化格局。这一因果链条揭示了算法治理的传导机制:监管并非直接作用于内容本身,而是通过改变平台激励来间接重塑内容生态,这种``间接治理''逻辑既是算法治理的优势所在——执行成本低、覆盖范围广,也是其局限的根源——非预期后果难以完全预判,多元目标之间存在潜在冲突。
