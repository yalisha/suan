\section{实证分析}

\subsection{描述性统计}

表\ref{tab:descriptive}报告了核心变量的描述性统计结果。样本涵盖2024年2月至2025年10月共608天的日度数据,对应253,227条话题观测值。就内容质量而言,社会类话题信息差指数均值达0.423,即超过四成的社会类话题标题呈现信息诱导特征;娱乐类话题该指标均值为0.318,相对较低。多样性指标方面,Shannon熵均值为0.782(标准差0.145),HHI均值为0.248(标准差0.062),表明热搜榜注意力分布存在较为明显的集中态势。注意力配置方面,社会类占据34.82\%的份额,娱乐核心类(含明星、游戏、综艺)占16.47\%,其中明星类占11.86\%。注意力密度方面,社会类单条话题平均在榜时长为8.45小时,娱乐核心类为7.23小时。

\begin{table}[htbp]
\centering
\caption{核心变量描述性统计}
\label{tab:descriptive}
\small
\begin{tabular}{lccccc}
\toprule
{\tabkai 变量} & {\tabkai 观测数} & {\tabkai 均值} & {\tabkai 标准差} & {\tabkai 最小值} & {\tabkai 最大值} \\
\midrule
\multicolumn{6}{l}{\textit{\tabkai 质量指标(H1)}} \\
$\var{InfoGap}${\tabkai (社会类)} & 211,842 & 0.423 & 0.494 & 0 & 1 \\
$\var{InfoGap}${\tabkai (娱乐类)} & 41,667 & 0.318 & 0.466 & 0 & 1 \\
$\var{Institutional}${\tabkai 占比(社会类)} & 608 & 0.267 & 0.089 & 0.103 & 0.512 \\
\midrule
\multicolumn{6}{l}{\textit{\tabkai 多样性指标(H2)}} \\
{\tabkai Shannon熵} & 608 & 0.782 & 0.145 & 0.412 & 1.134 \\
$\var{HHI}$ & 608 & 0.248 & 0.062 & 0.156 & 0.487 \\
{\tabkai 单条话题重复上榜次数} & 608 & 2.34 & 0.67 & 1.12 & 4.56 \\
{\tabkai 连续在榜时长(小时)} & 608 & 6.78 & 2.13 & 3.21 & 14.23 \\
\midrule
\multicolumn{6}{l}{\textit{\tabkai 注意力配置指标(H3)}} \\
$\var{Share}${\tabkai (社会类)} & 608 & 0.3482 & 0.0734 & 0.189 & 0.523 \\
$\var{Share}${\tabkai (娱乐核心类)} & 608 & 0.1647 & 0.0456 & 0.078 & 0.289 \\
$\var{Share}${\tabkai (明星类)} & 608 & 0.1186 & 0.0389 & 0.052 & 0.234 \\
$\var{Share}${\tabkai (游戏类)} & 608 & 0.0312 & 0.0145 & 0.011 & 0.078 \\
$\var{TotalHeat}${\tabkai (百万)} & 87 & 12.45 & 2.34 & 7.89 & 18.67 \\
\midrule
\multicolumn{6}{l}{\textit{\tabkai 注意力密度指标(H4)}} \\
{\tabkai 社会类总在榜时长(小时/日)} & 608 & 82.3 & 18.7 & 42.1 & 134.5 \\
$\var{Density}${\tabkai (社会类,小时)} & 608 & 8.45 & 2.13 & 4.56 & 15.23 \\
{\tabkai 娱乐核心类总在榜时长(小时/日)} & 608 & 38.6 & 12.4 & 16.8 & 72.3 \\
$\var{Density}${\tabkai (娱乐核心类,小时)} & 608 & 7.23 & 1.89 & 3.78 & 13.45 \\
\bottomrule
\end{tabular}
\begin{tablenotes}
\small
\item {\tabkai 注:观测数608对应日度数据,253,227对应话题级观测值,87对应周度数据;信息差指数为二元变量,均值代表信息诱导比例。}
\end{tablenotes}
\end{table}

\subsection{内容质量分化效应检验(H1)}

假设H1预测分级分类治理将导致社会类内容质量提升而娱乐类保持稳定。本文首先采用ITS方法检验社会类信息差指数的整体变化(H1a),继而检验娱乐类质量指标的稳定性(H1c)。

表\ref{tab:h1} Panel A报告了社会类信息差指数的ITS回归结果。列(1)显示,信息差指数在政策后即时下降0.0847($p<0.001$),降幅达20.0\%;趋势项系数为-0.0008($p=0.015$),信息诱导现象呈持续改善态势。列(2)显示,机构媒体占比即时上升0.0672($p<0.001$),增幅为25.2\%。Panel B显示,娱乐类信息差指数的即时效应为-0.0123($p=0.187$),趋势效应为-0.0002($p=0.624$),均不显著,表明娱乐类内容质量在监管前后未发生实质性变化,支持H1c。

\begin{table}[htbp]
\centering
\caption{内容质量分化效应(ITS回归)}
\label{tab:h1}
\small
\begin{tabular}{lcc}
\toprule
\multicolumn{3}{l}{\textit{\tabkai Panel A:社会类质量提升(H1a)}} \\
\midrule
{\tabkai 因变量} & (1) $\var{InfoGap}$ & (2) $\var{Institutional}${\tabkai 占比} \\
\midrule
$\var{Post}${\tabkai (即时效应)} & $-$0.0847$^{***}$ & 0.0672$^{***}$ \\
 & (0.0156) & (0.0089) \\
$\var{Trend}${\tabkai (趋势)} & 0.0003 & $-$0.0004 \\
 & (0.0003) & (0.0002) \\
$\var{Post}\times\var{Trend}${\tabkai (趋势效应)} & $-$0.0008$^{*}$ & 0.0006$^{**}$ \\
 & (0.0003) & (0.0002) \\
{\tabkai 常数项} & 0.423$^{***}$ & 0.267$^{***}$ \\
 & (0.0104) & (0.0059) \\
$N$ & 608 & 608 \\
$R^2$ & 0.524 & 0.618 \\
\midrule
\multicolumn{3}{l}{\textit{\tabkai Panel B:娱乐类质量稳定(H1c)}} \\
\midrule
{\tabkai 因变量} & \multicolumn{2}{c}{(3) $\var{InfoGap}$} \\
\midrule
$\var{Post}${\tabkai (即时效应)} & \multicolumn{2}{c}{$-$0.0123} \\
 & \multicolumn{2}{c}{(0.0093)} \\
$\var{Post}\times\var{Trend}${\tabkai (趋势效应)} & \multicolumn{2}{c}{$-$0.0002} \\
 & \multicolumn{2}{c}{(0.0004)} \\
$N$ & \multicolumn{2}{c}{608} \\
$R^2$ & \multicolumn{2}{c}{0.312} \\
\bottomrule
\end{tabular}
\begin{tablenotes}
\small
\item {\tabkai 注:}$^{***}p<0.001$,$^{**}p<0.01$,$^{*}p<0.05${\tabkai ;括号内为Newey-West标准误。}
\end{tablenotes}
\end{table}

质量分化效应的形成可从资源依赖视角加以理解。第一,差异化合规成本:社会类内容因其公共属性与舆情风险被纳入更严格的治理序列,平台为规避监管红线而大幅强化审核力度,合规成本显著上升。第二,资源依赖与挤出效应:高风险领域成为平台须优先满足的核心依赖,治理资源向社会类倾斜,对娱乐类形成挤出。第三,信号显示与合法性获取:平台借助社会类质量提升向监管者传递合规信号,而娱乐类因监管压力较小而维持既有质量水平。

\subsection{注意力去中心化效应检验(H2)}

假设H2预测算法治理将削弱头部话题的注意力垄断,提升热搜榜的多样性水平。本文以2024年11月12日为政策断点,采用ITS方法对Shannon熵、HHI、单条话题重复上榜次数及连续在榜时长四项指标进行检验,结果见表\ref{tab:h2}。

列(1)显示,Shannon熵在政策实施后即时上升0.146($p<0.001$),相对于基准均值0.782的增幅为18.7\%;趋势项系数为0.0012($p=0.002$),表明熵值在政策后呈持续上升态势。列(2)显示,HHI即时下降0.0292($p<0.001$),降幅为11.8\%,趋势项系数为-0.0003($p=0.042$),集中度持续走低。列(3)与列(4)分别显示,单条话题重复上榜次数下降0.547次(降幅23.4\%),连续在榜时长缩短2.15小时(降幅31.7\%),均在0.1\%水平显著。上述结果表明,算法治理通过限制单一话题的重复上榜频次与在榜时长,有效瓦解了头部垄断格局,推动了注意力分布的去中心化。

\begin{table}[htbp]
\centering
\caption{注意力去中心化效应(ITS回归)}
\label{tab:h2}
\small
\begin{tabular}{lcccc}
\toprule
 & (1) & (2) & (3) & (4) \\
{\tabkai 因变量} & {\tabkai Shannon熵} & $\var{HHI}$ & {\tabkai 重复上榜次数} & {\tabkai 连续在榜时长} \\
\midrule
$\var{Post}${\tabkai (即时效应)} & 0.146$^{***}$ & $-$0.0292$^{***}$ & $-$0.547$^{***}$ & $-$2.15$^{***}$ \\
 & (0.0234) & (0.0056) & (0.089) & (0.342) \\
$\var{Trend}${\tabkai (趋势)} & $-$0.0008 & 0.0002 & $-$0.0021 & 0.0156 \\
 & (0.0005) & (0.0001) & (0.0018) & (0.0089) \\
$\var{Post}\times\var{Trend}${\tabkai (趋势效应)} & 0.0012$^{**}$ & $-$0.0003$^{*}$ & $-$0.0089$^{***}$ & $-$0.0234$^{**}$ \\
 & (0.0005) & (0.0001) & (0.0024) & (0.0098) \\
{\tabkai 常数项} & 0.782$^{***}$ & 0.248$^{***}$ & 2.341$^{***}$ & 6.781$^{***}$ \\
 & (0.0156) & (0.0037) & (0.059) & (0.223) \\
\midrule
$N$ & 608 & 608 & 608 & 608 \\
$R^2$ & 0.742 & 0.681 & 0.635 & 0.587 \\
{\tabkai Newey-West标准误} & $\checkmark$ & $\checkmark$ & $\checkmark$ & $\checkmark$ \\
\bottomrule
\end{tabular}
\begin{tablenotes}
\small
\item {\tabkai 注:}$^{***}p<0.001$,$^{**}p<0.01$,$^{*}p<0.05${\tabkai ;括号内为Newey-West标准误;}$\var{Post}${\tabkai 为政策后虚拟变量(2024年11月12日后取1);}$\var{Trend}${\tabkai 为时间趋势(政策前归一化为0)。}
\end{tablenotes}
\end{table}

就机制而言,去中心化主要通过两条路径实现。其一,供给侧约束:《算法专项治理清单指引》明确规定平台``不得利用算法操纵热点话题'',对单一话题垄断榜单形成硬约束。其二,需求侧多样性激励:监管同时要求``保障信息内容多样性'',平台为规避单一话题过度曝光可能招致的监管风险,主动提升长尾话题的推荐权重。两条路径相互叠加,共同推动了注意力分布的均衡化。

\subsection{注意力再分配效应检验(H3)}

假设H3预测,在注意力零和约束下,对社会类的监管强化将引发注意力向娱乐类的系统性转移,形成``水床效应''。

表\ref{tab:h3}报告了各类别注意力份额的ITS回归结果。列(1)显示,社会类份额即时下降0.0316($p<0.001$),降幅为9.1\%,趋势项系数为-0.0008($p<0.001$),份额呈持续走低态势。列(2)显示,娱乐核心类份额即时上升0.0384($p<0.001$),增幅达23.3\%,趋势项系数为0.0006($p=0.003$)。进一步分解表明,明星类份额即时上升0.0361($p<0.001$),游戏类上升0.0112($p=0.002$),两者构成娱乐类扩张的主体。

\begin{table}[htbp]
\centering
\caption{注意力再分配效应(类别份额ITS回归)}
\label{tab:h3}
\small
\begin{tabular}{lccccc}
\toprule
{\tabkai 因变量} & (1) {\tabkai 社会类} & (2) {\tabkai 娱乐核心类} & (3) {\tabkai 明星类} & (4) {\tabkai 游戏类} & (5) {\tabkai 其他类} \\
\midrule
$\var{Post}${\tabkai (即时效应)} & $-$0.0316$^{***}$ & 0.0384$^{***}$ & 0.0361$^{***}$ & 0.0112$^{**}$ & $-$0.0068 \\
 & (0.0045) & (0.0038) & (0.0042) & (0.0035) & (0.0051) \\
$\var{Trend}${\tabkai (趋势)} & 0.0002 & $-$0.0001 & $-$0.0001 & 0.0000 & $-$0.0001 \\
 & (0.0001) & (0.0001) & (0.0001) & (0.0001) & (0.0001) \\
$\var{Post}\times\var{Trend}$ & $-$0.0008$^{***}$ & 0.0006$^{**}$ & 0.0005$^{**}$ & 0.0002$^{*}$ & 0.0002 \\
 & (0.0002) & (0.0002) & (0.0002) & (0.0001) & (0.0002) \\
{\tabkai 常数项} & 0.3482$^{***}$ & 0.1647$^{***}$ & 0.1186$^{***}$ & 0.0312$^{***}$ & 0.4871$^{***}$ \\
 & (0.0030) & (0.0025) & (0.0028) & (0.0023) & (0.0034) \\
\midrule
$N$ & 608 & 608 & 608 & 608 & 608 \\
$R^2$ & 0.693 & 0.728 & 0.715 & 0.542 & 0.398 \\
\bottomrule
\end{tabular}
\begin{tablenotes}
\small
\item {\tabkai 注:}$^{***}p<0.001$,$^{**}p<0.01$,$^{*}p<0.05${\tabkai ;括号内为Newey-West标准误。}
\end{tablenotes}
\end{table}

表\ref{tab:h3c}报告了周度总热度的断点检验结果。即时效应为0.0124($p=0.626$),趋势效应为-0.0089($p=0.734$),均不显著。这一结果表明,尽管注意力在不同类别间发生了显著再分配,但总量保持稳定,印证了注意力零和约束的存在。监管引发的是注意力的结构性调整而非总量变动。

\begin{table}[htbp]
\centering
\caption{周度总热度守恒检验(ITS回归)}
\label{tab:h3c}
\small
\begin{tabular}{lc}
\toprule
{\tabkai 因变量} & $\var{TotalHeat}${\tabkai (百万)} \\
\midrule
$\var{Post}${\tabkai (即时效应)} & 0.0124 \\
 & (0.0253) \\
$\var{Trend}${\tabkai (趋势)} & 0.0234 \\
 & (0.0189) \\
$\var{Post}\times\var{Trend}${\tabkai (趋势效应)} & $-$0.0089 \\
 & (0.0261) \\
{\tabkai 常数项} & 12.452$^{***}$ \\
 & (0.167) \\
\midrule
$N$ & 87 \\
$R^2$ & 0.234 \\
$p${\tabkai 值(}$\var{Post}${\tabkai )} & 0.626 \\
{\tabkai 结论} & {\tabkai 接受总量守恒假设} \\
\bottomrule
\end{tabular}
\begin{tablenotes}
\small
\item {\tabkai 注:括号内为Newey-West标准误;总热度为所有话题热度值的周度加总。}
\end{tablenotes}
\end{table}

上述结果验证了``水床效应''机制:监管提升社会类审核成本的同时,多样性约束限制了单一话题的垄断能力;平台在新约束下重新优化注意力配置,将受挤压的社会类配额再分配至审核成本较低的娱乐类,形成``社会类紧缩—娱乐类扩张''的跷跷板格局。这一发现揭示了算法治理的非预期后果:多样性提升与社会类质量改善的同时,社会类份额下降可能削弱公共议题的曝光机会,娱乐类份额上升可能重塑用户的信息消费结构。

\subsection{注意力密度分化效应检验(H4)}

假设H4预测,在注意力再分配过程中,社会类呈现``少而精''调整,娱乐类呈现``多且长''调整。

表\ref{tab:h4}报告了注意力密度指标的分组ITS回归结果。Panel A显示,社会类总在榜时长的即时效应为-12.3小时/日($p=0.210$),不显著;但时间份额即时下降0.0316($p<0.001$),单条话题平均在榜时长即时上升2.45小时($p<0.001$)。这表明社会类在总量稳定的情况下,通过压缩上榜话题数量、延长单条话题时长,实现了``少而精''的调整路径。Panel B显示,娱乐核心类总在榜时长即时上升18.7小时/日($p<0.001$),时间份额上升0.0384($p<0.001$),单条话题平均在榜时长上升1.89小时($p<0.001$)。娱乐类在总量、份额、密度三个维度均呈显著上升态势,形成``多且长''的调整格局。

\begin{table}[htbp]
\centering
\caption{注意力密度分化效应(分组ITS回归)}
\label{tab:h4}
\small
\begin{tabular}{lccc}
\toprule
\multicolumn{4}{l}{\textit{\tabkai Panel A:社会类``少而精''调整}} \\
\midrule
{\tabkai 因变量} & (1) {\tabkai 总在榜时长} & (2) $\var{Share}$ & (3) $\var{Density}$ \\
\midrule
$\var{Post}${\tabkai (即时效应)} & $-$12.3 & $-$0.0316$^{***}$ & 2.45$^{***}$ \\
 & (9.78) & (0.0045) & (0.456) \\
$\var{Post}\times\var{Trend}$ & $-$0.234 & $-$0.0008$^{***}$ & 0.0234$^{**}$ \\
 & (0.189) & (0.0002) & (0.0089) \\
$N$ & 608 & 608 & 608 \\
$R^2$ & 0.412 & 0.693 & 0.635 \\
\midrule
\multicolumn{4}{l}{\textit{\tabkai Panel B:娱乐核心类``多且长''调整}} \\
\midrule
{\tabkai 因变量} & (4) {\tabkai 总在榜时长} & (5) $\var{Share}$ & (6) $\var{Density}$ \\
\midrule
\textit{\tabkai 娱乐核心类} & & & \\
$\var{Post}${\tabkai (即时效应)} & 18.7$^{***}$ & 0.0384$^{***}$ & 1.89$^{***}$ \\
 & (3.45) & (0.0038) & (0.389) \\
$\var{Post}\times\var{Trend}$ & 0.156$^{**}$ & 0.0006$^{**}$ & 0.0156$^{*}$ \\
 & (0.067) & (0.0002) & (0.0078) \\
\textit{\tabkai 明星类} & & & \\
$\var{Post}${\tabkai (即时效应)} & 16.4$^{***}$ & 0.0361$^{***}$ & 1.67$^{***}$ \\
 & (2.89) & (0.0042) & (0.345) \\
\textit{\tabkai 游戏类} & & & \\
$\var{Post}${\tabkai (即时效应)} & 5.2$^{**}$ & 0.0112$^{**}$ & 0.84$^{*}$ \\
 & (1.95) & (0.0035) & (0.389) \\
$N$ & 608 & 608 & 608 \\
$R^2$ & 0.567 & 0.728 & 0.542 \\
\bottomrule
\end{tabular}
\begin{tablenotes}
\small
\item {\tabkai 注:}$^{***}p<0.001$,$^{**}p<0.01$,$^{*}p<0.05${\tabkai ;括号内为Newey-West标准误;总在榜时长单位为小时/日。}
\end{tablenotes}
\end{table}

密度分化揭示了平台在注意力再分配过程中的微观调整逻辑。社会类的``少而精''源于监管的双重约束:质量门槛淘汰低质内容(分母减少),多样性约束限制单一话题垄断但不禁止高质量话题获得较长时长(分子相对稳定),综合效应是密度上升。娱乐类的``多且长''源于水床效应的注意力溢出:平台通过增加娱乐类话题数量与单条话题时长来吸收从社会类溢出的配额,形成``蓄水池''效应。

\subsection{稳健性检验}

为验证上述结果的可靠性,本文进行了多项稳健性检验。

\subsubsection{DID稳健性检验与平行趋势检验(H1b)}

表\ref{tab:did}报告了H1b的DID回归结果。交互项系数为-0.0205($p<0.001$),表明政策后机构媒体相对自媒体的信息差净下降2.05个百分点,验证了平台``源头筛选''机制的有效性。

\begin{table}[htbp]
\centering
\caption{机构媒体相对优势DID检验(H1b)}
\label{tab:did}
\small
\begin{tabular}{lc}
\toprule
{\tabkai 因变量} & $\var{InfoGap}$ \\
\midrule
$\var{Post}\times\var{Institutional}${\tabkai (DID估计量)} & $-$0.0205$^{***}$ \\
 & (0.0032) \\
$\var{Institutional}$ & $-$0.0734$^{***}$ \\
 & (0.0089) \\
$\var{Post}$ & $-$0.0642$^{***}$ \\
 & (0.0123) \\
{\tabkai 话题固定效应} & $\checkmark$ \\
{\tabkai 时间固定效应} & $\checkmark$ \\
$N$ & 211,842 \\
$R^2$ & 0.457 \\
\bottomrule
\end{tabular}
\begin{tablenotes}
\small
\item {\tabkai 注:}$^{***}p<0.001${\tabkai ;括号内为聚类到话题层面的稳健标准误。}
\end{tablenotes}
\end{table}

图\ref{fig:parallel}报告了平行趋势检验结果。政策实施前(-9至-1期),交互项系数围绕零值波动且置信区间均包含零,表明机构媒体与自媒体在政策前不存在显著的差异化趋势,平行趋势假设成立。政策实施后(0至10期),系数显著为负,表明政策效应在实施后即刻显现并持续存在。

\begin{figure}[htbp]
\centering
\includegraphics[width=0.8\textwidth]{event_study_information_gap.png}
\caption{平行趋势检验:信息差指数的动态DID估计}
\label{fig:parallel}

\small
{\tabkai 注:图中展示了以政策实施前一期($t=-1$)为基准的动态DID估计系数及95\%置信区间。空心圆点表示各期DID系数估计值,虚线为95\%置信区间,实心方块表示基准期($t=-1$)。政策实施前各期系数均不显著异于零,支持平行趋势假设;政策实施后系数显著为负,表明政策效应持续存在。}
\end{figure}

\subsubsection{其他稳健性检验}

表\ref{tab:robust}汇总了其他稳健性检验结果。

\begin{table}[htbp]
\centering
\caption{稳健性检验汇总}
\label{tab:robust}
\small
\begin{tabular}{lcccc}
\toprule
{\tabkai 检验类型} & (1) {\tabkai 安慰剂} & (2) {\tabkai 替代指标} & (3) {\tabkai 断点敏感性} & (4) {\tabkai 异质稳健SE} \\
\midrule
{\tabkai 因变量} & {\tabkai Shannon熵} & {\tabkai Gini系数} & $\var{Share}${\tabkai (社会类)} & {\tabkai 各主效应} \\
{\tabkai 核心系数} & 0.0123 & $-$0.0456$^{***}$ & $[-0.0298, -0.0334]$ & {\tabkai 方向一致} \\
{\tabkai 标准误} & (0.0167) & (0.0089) & {\tabkai 均}$p<0.01$ & {\tabkai 显著性不变} \\
$p${\tabkai 值} & 0.728 & $<$0.001 & {\tabkai 全部}$<$0.01 & --- \\
{\tabkai 结论} & {\tabkai 通过} & {\tabkai 通过} & {\tabkai 通过} & {\tabkai 通过} \\
\bottomrule
\end{tabular}
\begin{tablenotes}
\small
\item {\tabkai 注:列(1)伪断点为2024年9月1日;列(2)以Gini系数替代Shannon熵;列(3)断点在2024年11月12日}$\pm$7{\tabkai 天窗口内移动;列(4)以异质稳健标准误替代Newey-West标准误。}
\end{tablenotes}
\end{table}

第一,安慰剂检验。以2024年9月1日为伪断点重新估计ITS模型,Shannon熵的即时效应为0.0123($p=0.728$),不显著,表明政策前不存在结构性断点。

第二,替代指标检验。以Gini系数替代Shannon熵和HHI,政策后即时下降0.0456($p<0.001$),与主要结论一致。

第三,断点敏感性检验。将断点在$\pm$7天窗口内移动,社会类份额的即时效应在$[-0.0298, -0.0334]$区间内,均在1\%水平显著。

第四,异质稳健标准误。以异质稳健标准误替代Newey-West标准误,核心结论均保持稳健。

此外,本文还进行了以下补充检验(结果备索):排除重大节假日样本、缩短样本期至政策前后各90天、以Poisson回归替代OLS估计计数型因变量,核心结论均保持稳健。

\subsection{小结}

综上,实证分析全面验证了四项核心假设。H1检验表明分级分类治理导致社会类质量显著提升而娱乐类保持稳定,DID估计和平行趋势检验进一步强化了因果识别;H2检验表明算法治理有效推动了注意力分布的去中心化(Shannon熵上升18.7\%,HHI下降11.8\%);H3检验表明在注意力零和约束下社会类份额下降3.16个百分点、娱乐类上升3.84个百分点,总热度守恒印证了``水床效应'';H4检验表明社会类呈``少而精''调整、娱乐类呈``多且长''调整。这些发现共同刻画了算法治理对平台生态系统的多维重构效应,验证了基于资源依赖理论的分析框架。
