算法推荐系统在优化用户体验的同时也引发了信息茧房、注意力垄断等问题,如何有效治理算法成为全球数字治理的核心议题。本文利用2024年11月12日中国清朗行动算法治理政策这一准自然实验,采用断点回归时间序列(ITS)与双重差分(DID)相结合的方法,基于微博热搜榜2024年2月至2025年10月共608天、超过253,227条观测值的高频数据,系统评估了算法治理对平台生态系统的重构效应。

研究发现:\textbf{第一},算法治理显著促进注意力分布去中心化,Shannon熵即时上升18.7\%(水平系数+0.146, p<0.001),HHI下降11.8\%(水平系数-0.0292, p<0.001),平台通过限制单一话题重复上榜次数(下降23.4\%)和连续在榜时长(缩短31.7\%)打破头部垄断格局。\textbf{第二},分级分类治理导致内容质量分化,社会类话题信息差指数水平下降0.561(p=0.036),官方媒体相对非官方媒体的信息差净下降0.0205(p<0.001),娱乐类话题质量指标无显著变化。\textbf{第三},在注意力零和约束下,监管引发社会类向娱乐类的系统性注意力再分配,社会类份额即时下降3.16个百分点(p<0.001)且趋势项持续走低,娱乐类份额上升3.84个百分点(明星类+3.61pp, 游戏类+1.12pp),周度总热度断点检验验证总量守恒(p=0.63),呈现明显的"水床效应"。\textbf{第四},注意力密度在类别间显著分化,社会类呈"少而精"(单条话题平均在榜时长显著上升),娱乐核心类呈"多且长"(总在榜时长、时间份额、平均在榜时长三指标均显著上升)。

本研究提出"元组织的元组织"和"水床效应"理论机制,拓展了平台治理理论边界。研究揭示了算法治理的复杂效应:监管成功实现多样性提升和社会类质量改善,但在注意力零和约束下引发非预期的注意力转移,社会类向娱乐类的大规模再分配可能削弱公共议题曝光度,形成"监管初衷"与"生态演化"之间的张力。研究为理解算法治理的系统性后果和非预期效应提供了新证据,对完善分级分类治理体系、构建"激励优质+动态调整+跨平台协同+用户赋权"的系统性框架具有重要政策启示。

\textbf{关键词}: 算法治理; 平台生态系统; 注意力经济; 断点回归时间序列; 水床效应

\textbf{JEL分类号}: D83(信息、知识与不确定性); L51(经济监管); L86(信息和互联网服务)