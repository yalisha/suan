\section{二、制度背景、文献回顾与研究假设

#\section{(一)理论基础:注意力作为稀缺资源的平台配置逻辑

\textbf{1. 注意力的零和性质与竞争机制\textbf{

在数字平台生态系统中,注意力构成了比信息本身更为稀缺的核心资源(Oestreicher-Singer & Sundararajan, 2012)。当内容供给呈指数级增长时,用户的认知带宽却受制于生理和时间约束而保持刚性。根据新浪微博2023年财报,用户日均使用时长约52分钟,其中热搜浏览时长约15–20分钟,这一有限的时间窗口构成了注意力资源的刚性约束。

注意力资源具有三个决定性特征。\textbf{第一,稀缺性\textbf{:用户日均使用时长存在上限,短期内缺乏弹性(Hosanagar et al., 2014)。\textbf{第二,竞争性\textbf{:注意力分配给某一话题必然减少其他话题的份额(Chen et al., 2024)。在任意时刻,热搜榜只显示50个固定位置,某一话题进入榜单必然挤出其他话题,形成典型的零和博弈格局。\textbf{第三,不可再生性\textbf{:注意力一旦消耗无法恢复,用户在某条娱乐话题上停留的时间永久性地从社会类话题的潜在配额中流失。

这种竞争关系在社交媒体平台上表现为注意力拥挤效应:当信息供给增加时,单位信息获得的注意力边际递减(Gelper et al., 2021)。设平台上有$N$条话题竞争用户注意力,用户总注意力预算为$\bar{A}
$,则第$i$条话题获得的注意力$A_i$满足注意力守恒约束:

$$
\sum_{i=1}^{N} A_i = \bar{A}
$$

这一零和约束在热搜榜体系中表现得尤为明显:榜内与榜外话题的阅读量呈数量级差异(Oestreicher-Singer & Sundararajan, 2012),榜单位置的瞬时稀缺性使得话题间竞争异常激烈。更为关键的是,热搜榜不仅是注意力的消费场所,更是注意力的生产场所。平台通过算法排序和榜单展示,将分散的用户浏览行为转化为可度量、可累积的社会关注度(Zhao et al., 2024)。上榜本身构成一种注意力增值过程,话题进入榜单后获得的注意力呈指数级增长,形成正反馈循环(Yi et al., 2022)。因此,热搜榜单的任何调整,都是对稀缺注意力资源分配权力的重新定义。

\textbf{2. 平台的约束优化模型\textbf{

基于注意力零和博弈的底层逻辑,平台内容分发系统可被抽象为一个多目标约束优化问题。传统上,社交媒体平台以最大化用户参与度为首要目标(Kleinberg et al., 2023; Cao et al., 2024)。设平台上第$i$类内容获得的注意力为$A_i$,其产生的用户参与度为$R_i(A_i)$,平台的目标函数可表示为:

$$
\max_{\{A_i\}} \quad \sum_i R_i(A_i)
$$

其中,$R_i(A_i)$通常为凹函数,体现了用户参与度的边际递减特征(He et al., 2024)。在监管介入前,平台仅面临基本的资源约束:

$$
\text{s.t.} \quad \sum_i A_i = \bar{A}
$$

这一优化问题的解往往导致注意力高度集中于少数头部内容,因为头部内容通常具有更高的边际参与度$R_i'(A_i)$。

算法治理行动根本性地改变了平台的约束集合。Huber等(2017)指出,外部监管通过引入新的治理约束,改变了平台生态系统价值创造与治理成本之间的权衡。具体而言,《算法专项治理清单指引》引入了三类新约束:

\textbf{第一,多样性下限约束\textbf{。设$D(\{A_i\})$为衡量注意力分布多样性的指标(如Shannon熵),监管要求:

$$
D(\{A_i\}) \geq D_{\min}^{\text{post}} > D_{\min}^{\text{pre}}
$$

这一约束直接限制了头部内容的注意力集中度,迫使平台为长尾内容预留基础曝光配额。

\textbf{第二,内容质量约束\textbf{。设$C_i(Q_i, R_i)$为维持第$i$类内容质量水平$Q_i$的治理成本,该成本函数依赖于内容的风险等级$R_i$。监管要求:

$$
Q_i \geq Q_i^{\min}(R_i), \quad \frac{\partial Q_i^{\min}}{\partial R_i} > 0
$$

对于风险较高的社会类内容,$Q_i^{\min}$显著提升,相应的审核成本$C_i$呈超线性增长(Dehling & Sunyaev, 2024)。

\textbf{第三,透明度与可追溯性约束\textbf{。监管要求平台留存榜单日志、公示排序机制,实质上增加了平台的合规运营成本。这些新增约束将平台的优化问题转化为:

$$
\max_{\{A_i, Q_i\}} \quad \sum_i R_i(A_i) - \sum_i C_i(Q_i, R_i)
$$

$$
\text{s.t.} \quad \sum_i A_i = \bar{A}, \quad D(\{A_i\}) \geq D_{\min}^{\text{post}}, \quad Q_i \geq Q_i^{\min}(R_i)
$$

这一约束优化框架揭示了算法治理的核心机制:\textbf{通过缩小平台的可行域,监管改变了最优解的结构,进而触发注意力资源的系统性重新配置\textbf{。

\textbf{3. 监管冲击下的约束条件变化\textbf{

2024年11月出台的《算法专项治理清单指引》构成了一次典型的外部制度冲击。与渐进式政策调整不同,该指引通过27项具体核验标准的发布与执行,在短时间内大幅提升了平台的合规门槛,触发了约束集合的离散式变化。

\textbf{多样性约束的硬化\textbf{。清单第7项明确要求平台具备防范信息茧房的举措,通过内容去重、打散干预等策略提升推送内容多样性。这将原本的软性建议转化为可量化、可核验的硬性约束,直接压缩了头部话题的注意力配额上限。从优化理论角度看,这相当于在原有可行域中强制剔除高集中度配置方案,迫使平台向更均匀的注意力分布移动。

\textbf{质量成本的非对称上升\textbf{。清单第11项要求健全异常账号监测机制,防范违规操纵榜单;第23项鼓励算法提升优质内容推送、识别违法网络谣言。结合中国既有的舆情管理实践,这些要求在不同内容类别间产生了差异化影响。对于具有舆论属性和社会动员能力的社会类话题,平台需要叠加多轮人工复核,引入专业背景人员进行敏感词审查和事实核验,导致单位内容的审核成本显著上升。相比之下,娱乐类话题的审核成本增幅相对较小,形成了监管强度的类别梯度。这种\textbf{分级分类治理\textbf{(risk-based tiered governance)构成了本文分析的制度核心。

#\section{(二)研究假说

\textbf{假说H1:注意力去中心化效应\textbf{

算法治理的首要目标是打破"赢者通吃"的注意力垄断格局。监管通过多样性约束的硬化,改变了平台的可行域。定义$N$个话题的注意力分布向量为$p = (p_1, p_2, \dots, p_N)$,并按降序排列$p_1 \ge p_2 \ge \dots \ge p_N$。监管措施通过内容去重与打散干预强制削减前$k$个头部话题的流量配额,并将其再分配至长尾话题。

数学上,这等价于监管后的分布$p^{\text{post}
}$被监管前的分布$p^{\text{pre}}$所Lorenz优势,记作$p^{\text{pre}} \succ p^{\text{post}}$。该关系满足洛伦兹优势条件:

$$
\begin{cases}
\displaystyle \sum_{i=1}^{m} p_i^{\text{pre}} \ge \sum_{i=1}^{m} p_i^{\text{post}}, & \forall m \in \{1, \dots, N-1\} \\
\displaystyle \sum_{i=1}^{N} p_i^{\text{pre}} = \sum_{i=1}^{N} p_i^{\text{post}} = 1
\end{cases}
$$

此不等式组表明,在任意截断点$m$,监管前的累积注意力集中度均不低于监管后。

\textbf{Shannon熵的舒尔凹性定理\textbf{指出,当注意力分布从高度集中转向相对均匀时,Shannon熵单调递增。Shannon熵定义为:

$$
H(p) = -\sum_{i=1}^{N} p_i \ln p_i
$$

由于熵函数$\phi(x) = -x \ln x$为严格凹函数($\phi''(x) = -1/x < 0$),根据优序理论,若$p^{\text{pre}} \succ p^{\text{post}}$,则必然蕴含$H(p^{\text{post}}) > H(p^{\text{pre}})$。

理论机制上,去中心化通过三条路径实现:\textbf{第一,目标函数重构\textbf{。监管引入拉格朗日乘数$\lambda$,新目标函数为$\mathcal{L}(p, \lambda) = \sum_i v_i p_i + \lambda (H(p) - \delta_{\text{post}})$,其中影子价格$\lambda > 0$代表平台为满足合规要求所必须放弃的边际商业收益。\textbf{第二,马太效应的累进调节\textbf{。《治理清单》中的去重与打散构成了双向调节机制,对高流量话题征收"关注度税",并转移支付给长尾话题,切断了"高曝光–高点击–更高曝光"的正反馈循环。\textbf{第三,长尾激活\textbf{。算法治理通过为长尾内容预留基础曝光配额,降低了话题进入榜单的边际成本,激活了原本沉默的长尾话题。

因此,假说H1可操作化为:

\textbf{H1a(熵值上升)\textbf{:$\mathbb{E}[H^{\text{post}}] > \mathbb{E}[H^{\text{pre}}]$

\textbf{H1b(集中度下降)\textbf{:$\mathbb{E}[\text{HHI}^{\text{post}}] < \mathbb{E}[\text{HHI}^{\text{pre}}]$,其中HHI定义为$\text{HHI} = \sum_{i=1}^{N} p_i^2$

两个指标从不同角度刻画同一现象:Shannon熵从信息论角度衡量分布的不确定性,HHI从经济学角度衡量市场的竞争结构,二者共同验证注意力去中心化的发生。

\textbf{假说H2:内容质量分化效应\textbf{

分级分类治理的核心在于对不同风险等级的内容施加差异化的质量约束。设平台内容审核的成本函数为$C(Q, R)$,其中$Q$为内容质量,$R$为内容的风险等级。社会类内容由于具有较强舆论属性和社会动员能力,其风险等级$R_S$显著高于娱乐类内容$R_E$,即$R_S > R_E$。

该成本函数满足\textbf{超模性结构\textbf{(supermodularity):质量与风险的交叉偏导数为正,$\frac{\partial^2 C}{\partial Q \partial R} > 0$,即在高风险情境下提升单位质量的边际成本显著高于低风险情境。形式化而言,采用广义Cobb–Douglas形式与指数风险项结合的成本函数:

$$
C(Q, R) = \alpha Q^\beta e^{\gamma R}, \quad \alpha, \beta, \gamma > 0
$$

在此结构下,对社会类内容而言,监管要求的质量下限$Q_{\min}^S$提升带来的成本增量远大于娱乐类:

$$
\Delta C_S = C(Q_{\min}^{S,\text{post}}, R_S) - C(Q_{\min}^{S,\text{pre}}, R_S) \gg \Delta C_E
$$

平台面临在满足监管约束前提下最小化运营成本的决策问题:

$$
\begin{aligned}
\min_{Q_S, Q_E} \quad & C(Q_S, R_S) + C(Q_E, R_E) \\
\text{s.t.} \quad & Q_S \ge \underline{Q}_S^{\text{post}}, \quad Q_E \ge \underline{Q}_E^{\text{post}}
\end{aligned}
$$

其中监管对高风险社会类内容的合规门槛提升更为显著:$\underline{Q}_S^{\text{post}} \gg \underline{Q}_S^{\text{pre}}$,而$\underline{Q}_E^{\text{post}} \approx \underline{Q}_E^{\text{pre}}$。根据KKT条件,最优解将在约束恰好绑定或内部点之间选择。结合成本函数的超模性,对于社会类,由于$\underline{Q}_S^{\text{post}}$设定了较高的质量下限,且在高风险水平$R_S$下边际成本随$Q$增长极快,平台缺乏过度合规的激励,因此最优解倾向于在约束边界取得角点解:$Q_S^* = \underline{Q}_S^{\text{post}} > Q_S^{\text{pre}}$。对于娱乐类,$R_E$较小使得边际成本曲线更为平缓,且$\underline{Q}_E^{\text{post}}$与监管前相比并未发生显著位移,平台在既有质量水平上即可满足最低合规要求:$Q_E^* \approx Q_E^{\text{pre}}$。

面对这一成本冲击,平台采取差异化策略。对于社会类内容,平台通过两条路径提升质量:\textbf{第一,源头筛选\textbf{。优先推送官方媒体(如人民日报、新华社)发布的内容,这些账号具有天然的权威性和可信度,审核成本较低且质量稳定。\textbf{第二,加强审核\textbf{。对非官方媒体账号实施更严格的事实核验和敏感词过滤,提升内容的信息含量,降低标题党和低质量信息的比例。

理论机制上,质量分化通过以下路径实现:

\textbf{机制一:基于风险的监管逻辑\textbf{。《算法专项治理清单指引》在异常账号监测与榜单操纵治理以及优化内容生态等模块中,对涉及公共事件、舆情风险和违法谣言的信息分发提出了更高要求。这体现了基于风险的监管逻辑:监管者根据潜在负外部性的大小分配治理资源与规范强度(Song et al., 2018)。具有较强公共属性和社会动员能力的社会类话题被视为高风险类别,一旦出现虚假信息或价值偏离,可能引发系统性舆情风险,因此被纳入更严格的治理序列。相对而言,娱乐类话题主要关联个体消费与情绪调节,其外部性更局限于低俗、侵权等局部问题,监管多采取守住底线的防御性姿态。

\textbf{机制二:资源依赖与挤出效应\textbf{。平台用于内容治理的预算、专业人力和算力等资源在短期内具有刚性约束。Huber等(2017)指出,当治理成本增量超过治理价值增量时,平台会调整其差异化策略以维持生态系统的整体效益。算法治理行动显著提高了社会类内容的边际合规成本后,高风险领域便成为组织必须优先满足的关键依赖对象。为了确保此类内容不触碰监管红线,平台不得不在资源配置上进行再平衡,将原本可用于优化长尾内容或娱乐内容体验的部分资源,转而集中投入到社会类话题的审核、风控与事后处置之中。这种以高风险领域为中心的资源重配,天然会产生对其他内容类别的挤出效应。

\textbf{机制三:信号显示与合法性获取\textbf{。在信息不对称的平台治理情境下,内容质量的提升构成平台向监管者发送合规信号的重要途径。Song等(2018)的研究表明,平台通过差异化治理策略来平衡不同利益相关者的诉求。对于高风险的社会类内容,平台倾向于采取更为显著的质量提升措施,如增加官方媒体占比、减少标题党内容,以向监管者展示其积极履行社会责任的姿态。然而,对于低风险的娱乐类内容,由于监管压力相对较小且质量提升的边际收益有限,平台更可能维持既有的质量水平。

\textbf{内容质量的操作化测度\textbf{。理论模型中的质量变量$Q_i$是一个多维潜变量,涵盖信息真实性、来源权威性、表达规范性等多个方面。本研究聚焦于表达规范性维度,特别是社交媒体平台普遍存在的"标题党"现象作为内容质量的核心测度。标题党(clickbait)是指通过夸张、煽动性或误导性标题吸引用户点击,但标题所承诺的信息与实际内容之间存在显著落差的现象。Loewenstein(1994)的信息差理论(information gap theory)为理解标题党提供了认知心理学基础:当个体意识到知识缺口时会产生好奇心,驱使其采取行动填补缺口。标题党正是利用这一机制,通过制造标题与内容之间的语义悬念来操纵用户的注意力分配(Golman & Loewenstein, 2016)。

本研究采用基于关键词词频统计的方法构建\textbf{信息差指数\textbf{。该指数基于Loewenstein(1994)的信息差理论,识别话题标题中制造好奇缺口的语言特征,包括疑问词(为何、为什么、怎么、什么、原因、真相等)、悬念词(曝、竟、暗示、去向、结局、神反转、揭秘、内幕、背后等)以及疑问标点(?、?)。只要标题中出现任一上述特征,即判定为存在信息差(赋值为1),否则为0。该二元测度直接捕捉了标题是否利用"知识缺口"策略吸引注意力。

因此,假说H2可操作化为:

\textbf{H2a(社会类质量提升)\textbf{:社会类话题的信息差指数(衡量标题党程度)在监管后显著下降

\textbf{H2b(官媒相对优势)\textbf{:社会类内部,官方媒体相对非官方媒体的信息差净下降(DID估计量显著为负)

\textbf{H2c(娱乐类稳定)\textbf{:娱乐类话题的质量指标在监管前后无显著变化

\textbf{假说H3:注意力再分配效应(水床效应)\textbf{

在注意力零和约束下,对某一内容类型的监管强化会引发注意力在不同类型间的系统性再分配,形成"水床效应"(waterbed effect)。这一效应源于三层机制的叠加作用。

\textbf{第一,成本冲击的替代效应\textbf{。监管大幅提升社会类内容的审核成本和质量门槛,同时多样性约束限制了头部社会类话题的重复上榜次数和连续在榜时长。设社会类和娱乐类内容的注意力份额分别为$A_S$和$A_E$,平台在新约束下重新优化注意力配置。

定义平台内容生态的总注意力预算为$\bar{A}$。在短期内,受限于用户认知负荷与日活跃用户规模的相对稳定,总注意力供给缺乏弹性,可视为外生常数。平台面临的约束优化问题可表示为:

$$
\max_{\{A_i\}} \ \Pi = \sum_{i \in \{S,E,O\}} \big[ R_i(A_i) - C_i(A_i,\xi_i) \big] + \mu\, D(A_S,A_E,A_O)
$$

$$
\text{s.t.} \quad \sum_{i} A_i \le \bar{A}, \quad A_i \ge 0
$$

其中,$i$代表内容类别($S=$社会,$E=$娱乐,$O=$其他);$R_i(A_i)$为凹收益函数($R_i'>0, R_i''<0$),反映用户参与度的边际递减;$C_i(A_i,\xi_i)$为凸成本函数,$\xi_i$为外生监管强度参数;$D(\cdot)$为多样性激励项,$\mu \ge 0$为多样性权重。注意力预算约束在最优解处通常是紧约束,即$\sum_i A_i^* = \bar{A}$。

对每一类内容$i$,若最优解处$A_i^*>0$,一阶条件(FOC)为:

$$
\frac{\partial R_i}{\partial A_i} - \frac{\partial C_i}{\partial A_i} + \mu \frac{\partial D}{\partial A_i} = \lambda, \quad i \in \{S,E,O\}
$$

其中$\lambda \ge 0$为注意力资源的影子价格,表示在最优点附近增加1单位注意力预算所带来的边际收益。这表明在最优点上,各类内容的"边际收益 − 边际成本 + 多样性边际贡献"在注意力空间内应当被拉平到同一水平$\lambda$。

现在考察监管冲击导致社会类监管强度参数$\xi_S$上升的情形。当$\xi_S$上升时,社会类的边际成本上升($\partial C_S / \partial A_S \uparrow$),为重新满足一阶条件,平台必须降低$A_S$以降低边际成本,同时增加$A_E$或$A_O$以吸收释放的注意力配额。由于娱乐类的边际成本保持稳定且用户粘性高,平台倾向于将受挤压的注意力配额再分配至审核成本较低的娱乐类内容,形成$\Delta A_S < 0$且$\Delta A_E > 0$的替代效应。

注意力守恒约束要求:

$$
\sum_i A_i = \bar{A} \quad \Rightarrow \quad \Delta A_S + \Delta A_E + \Delta A_{\text{other}} = 0
$$

当$\Delta A_S < 0$(社会类份额下降)时,必然存在其他类型$\Delta A_j > 0$作为对冲。实证上,我们预期娱乐类成为主要的注意力接收方,因为其审核成本低、风险等级低、用户粘性高。

\textbf{第二,用户偏好的显示效应\textbf{。当社会类话题的供给受到限制时,平台算法将更多地依赖用户的历史行为数据进行个性化推荐。对于偏好娱乐内容的用户群体,社会类话题曝光下降会释放其注意力预算,转而消费更多娱乐类内容。这种需求侧的调整与供给侧的成本优化相互强化,形成注意力的系统性转移。

\textbf{第三,平台的动态适应\textbf{。监管实施初期,平台可能通过缩短社会类话题的在榜时长、降低单条话题的重复上榜次数来满足多样性约束,导致社会类总在榜时长下降。为填补榜单位置的空缺,平台需要增加其他类型话题的上榜数量和时长,娱乐类因其内容供给丰富、审核成本低而成为首选。

综合三层机制,我们提出"水床效应"的核心预测:\textbf{在注意力零和约束下,对社会类内容的监管强化会导致注意力向娱乐类等低风险内容溢出,形成"社会类紧缩-娱乐类扩张"的跷跷板效应\textbf{。这一效应类似于经济学中的"水床效应"(waterbed effect),即对市场一端的压制会导致另一端的膨胀(Rahman et al., 2024)。

因此,假说H3可操作化为:

\textbf{H3a(社会类份额下降)\textbf{:社会类话题在热搜榜中的份额(按在榜时长或条数)在监管后显著下降

\textbf{H3b(娱乐类份额上升)\textbf{:娱乐类话题在热搜榜中的份额在监管后显著上升

\textbf{H3c(总量守恒)\textbf{:周度总热度(或总在榜时长)在监管前后无显著变化,验证注意力零和约束

\textbf{假说H4:注意力密度分化效应\textbf{

在注意力再分配过程中,不同类型内容的注意力密度(单条话题获得的平均注意力)会呈现差异化演化路径。注意力密度是理解生态系统微观调整机制的关键指标,它揭示了在总量重新分配的同时,单位内容获得注意力的强度变化(Oestreicher-Singer & Sundararajan, 2012)。我们定义类内注意力密度为:

$$
\text{Density}_i = \frac{\text{Total Attention}_i}{\text{Number of Topics}_i}
$$

该指标分解了总注意力变化的两个维度:\textbf{广度\textbf{(话题数量)与\textbf{深度\textbf{(单条话题强度)。密度上升意味着虽然话题总数可能减少,但每条话题获得了更强的注意力聚焦;密度下降则意味着注意力在更多话题间稀释。

对于社会类内容,监管的双重作用导致"少而精"调整:\textbf{第一,数量收缩\textbf{。多样性约束限制单一话题重复上榜次数(如"某明星"不能连续霸榜),质量门槛抬升淘汰低质量内容,导致上榜话题数量减少(分母$\downarrow$)。\textbf{第二,质量提升\textbf{。官媒权重增加和内容质量改善可能提升单条话题的用户停留时长和深度阅读率,分子相对稳定或小幅下降。综合效应是社会类的注意力密度上升或保持稳定,体现了平台从"量变"向"质变"的策略转移。

这种"少而精"调整符合平台在资源约束下的理性选择:当审核成本上升时,平台倾向于减少社会类话题的供给数量,但通过提升质量来维持用户参与度,避免因内容数量骤减而导致用户流失。此外,多样性约束要求平台打破单一话题的垄断格局,但并不禁止高质量话题获得较长的在榜时长,因此社会类可能通过"更少但更优质"的话题来维持总注意力份额。

对于娱乐类内容,注意力再分配导致"多且长"调整:\textbf{第一,数量扩张\textbf{。平台为填补社会类留下的榜单空缺,增加娱乐类话题的上榜数量,特别是明星类和游戏类话题的曝光频次显著提升(分母$\uparrow$)。\textbf{第二,总量上升\textbf{。由于水床效应,社会类份额下降释放的注意力配额主要流向娱乐类,导致娱乐类总在榜时长和总热度显著上升(分子$\uparrow\uparrow$)。由于分子增幅大于分母增幅(即总注意力增长速度快于话题数量增长速度),娱乐类的注意力密度上升,单条话题获得更多曝光和停留时长。

这种"多且长"调整反映了娱乐类内容在监管后成为平台注意力配置的"蓄水池"。由于娱乐类的审核成本低、用户粘性高、内容供给丰富,平台可以在不显著增加合规风险的前提下,通过增加娱乐类话题数量和单条话题时长来吸收从社会类溢出的注意力,维持榜单的活跃度和用户参与度。

因此,假说H4可操作化为:

\textbf{H4a(社会类密度稳定或上升)\textbf{:社会类单条话题的平均在榜时长在监管后保持稳定或上升,体现"少而精"调整

\textbf{H4b(娱乐类密度上升)\textbf{:娱乐类单条话题的平均在榜时长、总在榜时长、时间份额在监管后显著上升,体现"多且长"调整

综上所述,本文构建了一个整合平台治理理论、注意力经济理论和双边网络理论的分析框架,提出了四大核心假说:H1预测注意力去中心化,H2预测内容质量分化,H3预测注意力再分配(水床效应),H4预测注意力密度分化。这四大假说共同刻画了算法治理对平台生态系统的多维重构效应,为后续实证分析提供了理论指引。
